\section{CMake}

\href{https://cmake.org}{CMake} is a tool to manage the software build process. 
Originally, CMake was designed as a generator for various dialects of Makefile, today CMake generates modern buildsystems such as Ninja as well as project files for IDEs such as Visual Studio and Xcode. \\
Most of the information bellow is adapted from the main documentation: 
\begin{center}\href{https://cmake.org/cmake/help/latest/}{https://cmake.org/cmake/help/latest/} \end{center}
Also a GUI tool called \bftt{cmake-gui} is available, but its usage is not covered in this manual. \\
Finally please note that in the following I will only consider basic, targeted, usage of \href{https://cmake.org}{CMake}, that is a complexe tool with much more capabilities. \\[0.25cm]
As for the GNU Autotools, \href{https://en.wikipedia.org/wiki/Pkg-config}{pkg-config} will be used extensively thereafter.  
Before going further into this guide, and if you want to try to build and package your pogram using CMake, you will have to install the following tools: 
\begin{itemize}
\item For Windows and OSX please refer to the corresponding websites.
\item For Linux you can quite conveniently use the command line as follow:
\begin{itemize}
\item Red Hat based Linux (using the \bftt{dnf} command):
\begin{scriptii}
\fprompt{~} sudo \bftt{dnf} install cmake
\fprompt{~} sudo \bftt{dnf} install pkg-config gcc gcc-gfortran
\end{scriptii}
\item Debian based Linux (using the \bftt{apt} command):
\begin{scriptii}
\uprompt{~} sudo \bftt{apt} install cmake
\uprompt{~} sudo \bftt{apt} install pkg-config gcc gfortran
\end{scriptii}
\end{itemize}
\end{itemize}
To learn more about software installation on Linux check out my \href{https://www.ipcms.fr/wp-content/uploads/2021/11/linux.pdf}{Linux tutorial}. \\
To learn more about scripting and the usage of the command line check out my \href{https://www.ipcms.fr/wp-content/uploads/2021/05/bash.pdf}{Bash tutorial}. 

\subsection{Example project}

In the next sections the example project will be similar to the case of the GNU autotools: 
{\footnotesize{
\begin{script}
\fprompt{~/program} ls -lh
-rw-r--r--. 1 user group 2,8K 24 mars  11:24 ChangeLog
-rw-r--r--. 1 user group  34K 24 mars  11:24 COPYING
drwxr-xr-x. 2 user group 4,0K 24 mars  11:24 \btt{data}
drwxr-xr-x. 2 user group 4,0K 24 mars  11:24 \btt{metadata}
drwxr-xr-x. 4 user group 4,0K 24 mars  11:24 \btt{pixmaps}
-rw-r--r--. 1 user group 4,8K 24 mars  11:24 README
drwxr-xr-x. 2 user group 4,0K 24 mars  11:24 \btt{src}
\end{script}
}}
\vspace{-0.25cm}
\\
\noindent The top directory contains 4 subfolders:
\begin{itemize}
\item \bftt{src}:
{\footnotesize{
\begin{scripti}
\fprompt{~/program} ls src
file-1.f90    file-2.f90    gui.c    main.c    mod.f90
\end{scripti}
}}
\vspace{-0.75cm} \item \bftt{data}:
{\footnotesize{
\begin{scripti}
\fprompt{~/program} ls data
file-1.dat    file-2.dat
\end{scripti}
}}
\vspace{-0.75cm}
\item \bftt{pixmaps}:
{\footnotesize{
\begin{scripti}
\fprompt{~/program} ls pixmaps/*
pixmaps/pix-3.dat

pixmaps/\btt{pix-1}:
pix-1-a.dat    pix-1-b.dat

pixmaps/\btt{pix-2}:
pix-2-a.dat    pix-2-b.dat
\end{scripti}
}}
\vspace{-0.75cm} \item \bftt{metadata} (this is used for Freedesktop integration, see [Sec.~\ref{rpmpost}]):
{\footnotesize{
\begin{scripti}
\fprompt{~/program} ls metadata
com.program.www.appdata.xml    program.desktop    program-mime.xml

metadata/\btt{icons}:
program.svg    program-project.svg    program-workspace.svg
\end{scripti}
}}
\end{itemize}

\subsection{The file: \bftt{CMakeLists.txt}} 

The file \bftt{CMakeLists.txt} is a configuration file that defines:  
\begin{itemize}
\item The rules to configure your project
\item The rules to build your project
\item The rules to install your project
\end{itemize}
In the next pages I will browse briefly the differents parts of the file \bftt{CMakeLists.txt}, however my approach will remain simple. 
For detailed information check the \href{https://cmake.org/cmake/help/latest/index.html}{official user manual} \\[0.25cm]
To start working your CMake package you will need first to create a directory to work in:
\begin{script}
\fprompt{~} mkdir program
\fprompt{~} cd program
\end{script}
\\[-0.25cm]
\noindent Then simply use a text editor to work on this file and insert instructions regarding configuration, building and installation of your program:
\begin{script}
\fprompt{~/program} \bftt{vi} \rtt{CMakeLists.txt}
\end{script}
\\[-1.5cm]

\subsubsection*{CMake version information}

The file \bftt{CMakeLists.txt} always starts by the minimum CMake version to use: 
\begin{script}
\confa{cmake_minimum_required} (\bftt{VERSION} 3.10)
\end{script}
\\[-1.5cm]

\subsubsection*{Project name, version and programmation languages}
\label{cmake_pname}

Describe project information using the \bftt{project} command:
\begin{script}
\confb{project}{\gtt{prog} \bftt{VERSION} 1.2.12}
\confb{project}{\gtt{prog} \bftt{DESCRIPTION} \magenta{"A tool box"}}
\confb{project}{\gtt{prog} \bftt{HOMEPAGE_URL} \magenta{"https://www.program.com"}}
\confb{project}{\gtt{prog} \bftt{LANGUAGES} C Fortran}
\end{script}
\\[-0.25cm]
\noindent Where \gtt{prog} is the project name, and is the keyword that will used thereafter to associate CMake actions to the project. \\ 
CMake will split the \bftt{VERSION} command in:
\begin{center}\texttt{MAJOR\_VERSION.MINOR\_VERSION.PATCH\_VERSION} \end{center}
The \bftt{LANGUAGES} option allows to check for compilers, however other dependencies might be needed (see bellow). \\[0.25cm]
For more information see the CMake \confb{project}{} command related documentation \href{https://cmake.org/cmake/help/latest/command/project.html\#command:project}{here}.

\subsubsection*{Dependencies}
\label{cmake_deps}

In this manual I will only focus on the \bftt{pkg-config} way of doing things, other possibilites are available and documented in the official CMake documentation: 
\begin{center}\href{https://cmake.org/cmake/help/latest/command/find\_package.html\#find-package}{https://cmake.org/cmake/help/latest/command/find\_package.html\#find-package}\end{center}
Check for package(s) using the \confb{find\_package}{} and \confb{pkg\_check\_modules}{} commands: 
\begin{script}
\confb{find_package}{PkgConfig \bftt{REQUIRED}}

\comm{Checking for gtk+3.0:}
\confb{pkg_check_modules}{\magenta{GTK3} \bftt{REQUIRED} IMPORTED_TARGET \dgtt{gtk+-3.0}}

\comm{Checking for libxml-2.0:}
\confb{pkg_check_modules}{\magenta{LIBXML2} \bftt{REQUIRED} IMPORTED_TARGET \dgtt{libxml-2.0}}

\comm{Checking for libavcodec, and other FFMPEG based libraries:}
\confb{pkg_check_modules}{\magenta{LIBAVUTIL} \bftt{REQUIRED} IMPORTED_TARGET \dgtt{libavutil}}
\confb{pkg_check_modules}{\magenta{LIBAVCODEC} \bftt{REQUIRED} IMPORTED_TARGET \dgtt{libavcodec}}
\confb{pkg_check_modules}{\magenta{LIBAVFORMAT} \bftt{REQUIRED} IMPORTED_TARGET \dgtt{libavformat}}
\confb{pkg_check_modules}{\magenta{LIBSWSCALE} \bftt{REQUIRED} IMPORTED_TARGET \dgtt{libswscale}}

\comm{Checking for epoxy:}
\confb{pkg_check_modules}{\magenta{EPOXY} \bftt{REQUIRED} IMPORTED_TARGET \dgtt{epoxy}}
\end{script}
\\[-0.25cm]
\noindent Where keywods in pink define variable names for the CMake project, eg. \texttt{\magenta{GTK3}}, 
and keywords in dark green, eg. \dgtt{gtk3}, are the associated to \bftt{pkg-config} libraries (see Sec.~\ref{clibs}). \\ 
You will also need to link the project to the required libraries, but only after the build target is declared, see page~\pageref{clink}.

\subsubsection*{Compiler options}

To define specific compiler flags use the \confb{set}{} command with the \confa{CMAKE\_}\bftt{<LANG>}\confa{\_FLAGS}
argument, \bftt{<LANG>} being the target programming language:
\begin{script}
\confb{set}{\confa{CMAKE_C_FLAGS} \magenta{"-O3 -fopenmp"}}
\confb{set}{\confa{CMAKE_Fortran_FLAGS} \magenta{"-O3 -fopenmp"}}
\end{script}

\subsubsection*{Preprocessor variable definitions}

To define compiler variables use the \confb{add\_compile\_definitions}{} instruction:
\begin{script}
\confb{add_compile_definitions}{OPENMP}
\confb{add_compile_definitions}{PACKAGE_LOGO=\magenta{"pixmaps/logo.png"}}
\end{script}
\\[-0.25cm]
\noindent On a GNU compiler command line, the previous instructions would be translated in:
\begin{script}
\bftt{gcc} \dgtt{-D}OPENMP \dgtt{-D}PACKAGE_LOGO=\magenta{"pixmaps/logo.png"}
\end{script}

\subsubsection*{Declaring the project sources}

In CMake you declare sources in list of file(s), that your refer to using a variable name.
\begin{itemize}
\item Recursively, not recommended for a distribution purposes:
{\footnotesize{
\begin{scripti}
\confb{file}{GLOB_RECURSE \bftt{C_SOURCES} RELATIVE \varc{CMAKE_SOURCE_DIR} \magenta{"src/*.c"}}
\confb{file}{GLOB_RECURSE \bftt{F_SOURCES} RELATIVE \varc{CMAKE_SOURCE_DIR} \magenta{"src/*.f90"}}
\end{scripti}
}} \\[-0.5cm]
\noindent \bftt{C\_SOURCES} is the name of a variable that will contain the list of files with the extension \texttt{\magenta{"*.c"}} that CMake can find recursively in the source directory \texttt{\magenta{"src"}}. \\
Similarly \bftt{F\_SOURCES} is the name of a variable that will contain the list of files with the extension \texttt{\magenta{"*.f90"}} that CMake can find recursively in the source directory \texttt{\magenta{"src"}}.
\item Folder by folder, then you need to include any sub-folder that contains source file(s):
{\footnotesize{
\begin{scripti}
\confb{file}{GLOB \bftt{C_SRC} RELATIVE \varc{CMAKE_SOURCE_DIR} \magenta{"src/*.c"}}
\confb{file}{GLOB \bftt{F_SRC} RELATIVE \varc{CMAKE_SOURCE_DIR} \magenta{"src/*.f90"}}
\end{scripti}
}} \\[-0.5cm]
\noindent \bftt{C\_SRC} is the name of a variable that will contain the list of files with the extension \texttt{\magenta{"*.c"}} that CMake can find in the source directory \texttt{\magenta{"src"}} only. \\
Similarly \bftt{F\_SRC} is the name of a variable that will contain the list of files with the extension \texttt{\magenta{"*.f90"}} that CMake can find in the source directory \texttt{\magenta{"src"}} only. 
\end{itemize}

\newpage
\subsubsection*{Declaring the project building process}

How to build the project is declared using the \confb{add\_executable}{} instruction:
\begin{script}
\confb{add_executable}{\gtt{prog} \varc{C_SRC} \varc{F_SRC}}
\end{script}
\\[-0.25cm]
\noindent Where \gtt{prog} is the project name (see page~\pageref{cmake_pname}), followed by the list of sources, here C and Fortran90 sources files declared previously and listed in the \varc{C\_SRC} and \varc{F\_SRC} variables. 

\subsubsection*{Declaring project headers include directories}

When the buidling instructions have been specified using the \confb{add\_executable}{} instruction, then it possible to declare project headers include directory using: 
\begin{script}
\confb{target_include_directories}{\gtt{prog} \bftt{PUBLIC} src}
\end{script} \\[-0.5cm]
\noindent Where \gtt{prog} is the project name (see page~\pageref{cmake_pname}), followed by a space separated list of directory(ies) to look into to search for header files. 

\subsubsection*{Linking to external libraries}
\label{clink}
Again only focusing on the \bftt{pkg-config} way of doing things, but other options are available:
\begin{center}\href{https://cmake.org/cmake/help/latest/command/target\_link\_libraries.html}{https://cmake.org/cmake/help/latest/command/target\_link\_libraries.html} \end{center}
To link the project to libraries use the \confb{target\_link\_libraries}{} instructions:
\begin{script}
\confb{target_link_libraries}{\gtt{prog} \bftt{PUBLIC} PkgConfig::\magenta{GTK3}}
\confb{target_link_libraries}{\gtt{prog} \bftt{PUBLIC} PkgConfig::\magenta{LIBXML2}}
\confb{target_link_libraries}{\gtt{prog} \bftt{PUBLIC} PkgConfig::\magenta{LIBAVUTIL}}
\confb{target_link_libraries}{\gtt{prog} \bftt{PUBLIC} PkgConfig::\magenta{LIBAVCODEC}}
\confb{target_link_libraries}{\gtt{prog} \bftt{PUBLIC} PkgConfig::\magenta{LIBAVFORMAT}}
\confb{target_link_libraries}{\gtt{prog} \bftt{PUBLIC} PkgConfig::\magenta{LIBSWSCALE}}
\confb{target_link_libraries}{\gtt{prog} \bftt{PUBLIC} PkgConfig::\magenta{EPOXY}}
\end{script}
\\[-0.25cm]
\noindent Where \gtt{prog} is the project name, see page~\pageref{cmake_pname}, and the keywords in pink in the \confb{target\_link\_libraries}{} instructions, 
refers to the keywords used in the \confb{pkg\_check\_modules}{} instructions, eg. \texttt{\magenta{GTK3}}, see page~\pageref{cmake_deps}.

\newpage
\subsubsection*{Installation using CMake}

To generate installation rules for your project use the \confb{install}{} instruction:
{\footnotesize{
\begin{script}
\comm{First import the GNU installation directory variables}
\confb{include}{\bftt{GNUInstallDirs}}

\comm{To install the main binary file:}
\confb{install}{\bftt{PROGRAMS} prog \bftt{DESTINATION} \varc{CMAKE_INSTALL_BINDIR}}

\comm{To install a directory, including all its content, recursively:}
\confb{install}{\bftt{DIRECTORY} data
\bftt{        DESTINATION} \varc{CMAKE_INSTALL_DATADIR}/prog
\bftt{        PATTERN} \magenta{"data/*"}}
\confb{install}{\bftt{DIRECTORY} pixmaps
\bftt{        DESTINATION} \varc{CMAKE_INSTALL_DATADIR}/prog
\bftt{        PATTERN}  \magenta{"pixmaps/*"}}
\confb{install}{\bftt{DIRECTORY} metadata/icons
\bftt{        DESTINATION} \varc{CMAKE_INSTALL_DATADIR}/pixmaps
\bftt{        PATTERN} \magenta{"metadata/icons/*.svg"}}

\comm{To install specific files:}
\confb{install}{\bftt{FILES} metadata/com.program.www.appdata.xml 
\bftt{        DESTINATION} \varc{CMAKE_INSTALL_DATADIR}/metainfo}
\confb{install}{\bftt{FILES} metadata/program-mime.xml
\bftt{        DESTINATION} \varc{CMAKE_INSTALL_DATADIR}/mime/packages}
\comm{To install ChangeLog}
\confb{install}{\bftt{FILES} ChangeLog \bftt{DESTINATION} \varc{CMAKE_INSTALL_DOCDIR}}
\comm{To install license information}
\confb{install}{\bftt{FILES} COPYING \bftt{DESTINATION} \varc{CMAKE_INSTALL_DOCDIR}}
\end{script}
}}

\subsubsection*{Post-installation script}

To use apply post-installation configuration, including system integration, an option is to use a script call by the \confb{install}{} command with 
the \bftt{CODE} argument: 
{\footnotesize{
\begin{script}
\comm{Creating a variable that contains the installation directory:}
\confb{set}{\bftt{INSTALL_DIR} "\varc{CMAKE_INSTALL_PREFIX}/\varc{CMAKE_INSTALL_DATADIR}"}

\comm{Runnng the post installation script:}
\confb{install}{\bftt{CODE} \magenta{"}\confb{execute_process}{\bftt{COMMAND} \magenta{./post-install.sh} \varc{INSTALL_DIR}}\magenta{"}}
\end{script}
}}
\\[-0.5cm]
This will execute the command \texttt{./post-install.sh} which is a script containing the post-installation instructions. 
It is required to transmit to the script the installation directory selected by CMake, this is done using the variable \varc{INSTALL\_DIR}
in argument of the script. 
Note that you can define variable, eg. \varc{INSTALL\_DIR}, using the \confb{set}{} instruction. \\
An example of content for the script \texttt{post-install.sh} is provided in appendix~\ref{poscript}. \\[0.25cm]
A complete example of the file \bftt{CMakeLists.txt} is provided in appendix~\ref{cmakelist}.

\subsection{Configuring, building and installing a CMake package}

To configure, build and install a CMake package use the \bftt{cmake} command. \\
In the following I will consider that the file \bftt{CMakeLists.txt} was completed: 
{\footnotesize{
\begin{script}
\fprompt{~/program} ls -lh
-rw-r--r--. 1 user group 2,8K 24 mars  12:25 ChangeLog
-rw-r--r--. 1 user group 5,3K 24 mars  12:25 \rtt{CMakeLists.txt}
-rw-r--r--. 1 user group  34K 24 mars  12:25 COPYING
drwxr-xr-x. 2 user group 4,0K 24 mars  12:25 \btt{data}
drwxr-xr-x. 2 user group 4,0K 24 mars  12:25 \btt{metadata}
drwxr-xr-x. 4 user group 4,0K 24 mars  12:25 \btt{pixmaps}
-rw-r--r--. 1 user group 4,8K 24 mars  12:25 README
drwxr-xr-x. 2 user group 4,0K 24 mars  12:25 \btt{src}
\end{script}
}}\\[-0.25cm]
\noindent The first step of the process is to create a temporary directory to work on to build the package:
{\footnotesize{
\begin{script}
\fprompt{~/program} \bftt{mkdir} build
\fprompt{~/program} \bftt{cd} build
\end{script}
}}\\[-0.25cm]
\noindent In CMake tutorials this directory is usually named \bftt{build}. 

\subsubsection*{Configuring}

To configure use \bftt{cmake}:
\begin{itemize}
\item Without option
\item Followed by the name of the location of the file \bftt{CMakeLists.txt}
\end{itemize}
\vspace{-1cm}
\begin{script}
\fprompt{~/program/build} \bftt{cmake} \dgtt{../}
\end{script}
\\[-0.5cm]
\noindent The single argument of the command is the location of the file \bftt{CMakeLists.txt}, \\
in this example case "\dgtt{../} $=$ upper directory". \\
The result of this command is the creation of several files and a directory:
\begin{script}
\fprompt{~/program/build} ls
CMakeCache.txt  \btt{CMakeFiles}  cmake_install.cmake  \rtt{Makefile}
\end{script}
\\[-0.5cm]
\noindent The new files and repository are as followed:
\begin{itemize}
\item \texttt{CMakeCache.txt}: list of environment variables used to build and install your project. 
\item \btt{CMakeFiles}: a directory that will be used to build your project. 
\item \texttt{cmake\_install.cmake}: the installation script generated by CMake for your project.
\item \rtt{Makefile}: the Makefile generated by CMake for your project.
\end{itemize}

\subsubsection*{Building}

To build use: 
\begin{itemize}
\item \bftt{cmake}:
\begin{itemize}
\item With the \rtt{-{-}build} option
\item Followed by the name of the location of the file \bftt{CMakeLists.txt}
\end{itemize}
\vspace{-1cm}
\begin{scripti}
\fprompt{~/program/build} \bftt{cmake} \rtt{--build} \dgtt{../} \btt{-j} \dctt{12}
\end{scripti} 
\\[-0.75cm]
\noindent The first option of the command is to request to build the project, it is followed by the is the location of the file \bftt{CMakeLists.txt}, in this example case "\dgtt{.} $=$ current directory", and optionnaly by other options, in this example a parallel build (\btt{-j}) using \dctt{12} threads. 
\item \bftt{make}
\begin{itemize}
\item Invoke make in the directory where the \bftt{Makefile} is located
\item With the project name as target, in this example: \rtt{prog}
\end{itemize}
\vspace{-0.75cm}
\begin{scripti}
\fprompt{~/program/build} \bftt{make} \rtt{prog} \btt{-j} \dctt{12}
\end{scripti} 
\end{itemize}
\vspace{-1cm}
\subsubsection*{Installing}

To install use:
\begin{itemize}
\item \bftt{cmake}:
\begin{itemize}
\item With the \rtt{-{-}install} option
\item Followed by the name of the location of the file \bftt{CMakeLists.txt}
\end{itemize}
\vspace{-1cm}
\begin{scripti}
\fprompt{~/program} \bftt{cmake} \rtt{--install} \dgtt{.} \btt{--prefix} \dctt{\$HOME}
\end{scripti}
\\[-0.75cm]
\noindent The first option of the command is to request to install the project, it is followed by the is the location of the file \bftt{CMakeLists.txt}, in this example case "\dgtt{.} $=$ current directory", and optionnaly by other options in this example to override the default installation directory using the \btt{-{-}prefix} option followed by the new installation directory. 
\item \bftt{make}:
\begin{itemize}
\item Invoke make in the directory where the \bftt{Makefile} is located
\item With the \rtt{-{-}install} option
\end{itemize}
\vspace{-0.75cm}
\begin{scripti}
\fprompt{~/program/build} \gtt{sudo} \bftt{make} \rtt{--install}
\end{scripti} 
\end{itemize}
\subsection{Packaging with CPack}

\href{https://cmake.org/cmake/help/book/mastering-cmake/chapter/Packaging\%20With\%20CPack.html}{CPack} is a cross-platform software packaging tool distributed with CMake. 
It can be used with or without CMake, to use CPack with CMake simply add the following instruction in the file \bftt{CMakeLists.txt}:
\begin{script}
\confb{include}{CPack}
\end{script}
\\[-0.5cm]
\noindent Then after the configuraton stage, new files are created in the active directory: 
\begin{script}
\fprompt{~/program/build} ls 
CMakeCache.txt        \btt{CMakeFiles}                 cmake_install.cmake  
\bftt{CPackConfig.cmake}  \bftt{CPackSourceConfig.cmake}  Makefile
\end{script}\\[-0.5cm]
The files \bftt{CPackConfig.cmake} and \bftt{CPackSourceConfig.cmake} can be used by CPack to easily prepare packages to ship the binary version of your project, or the sources: 
%{\footnotesize{
\begin{script}
\fprompt{~/program/build} \bftt{cpack} \rtt{-C} \dgtt{CPackConfig.cmake} \rtt{-G} \dgtt{TBZ2}
Pack: Create package using TBZ2
CPack: Install projects
CPack: - Run preinstall target for: prog
CPack: - Install project: prog [CPackConfig.cmake]
CPack: Create package
CPack: - package: ~/program/build/prog-0.1.1-Linux.tar.bz2 generated.
\fprompt{~/program/build} ls *.bz2
\rtt{prog-0.1.1-Linux.tar.bz2}
\end{script}
%}}
\\[-0.5cm]
CPack creates a package using: 
\begin{itemize}
\item A configuration file designated using the option \rtt{-C} (or \rtt{-{-}config})
\item The type of package using the option \rtt{-G}
\end{itemize}

\subsection{Difference with the GNU autotools}

The main difference between the GNU autotools introduced in the first part of the chapter, and CMake, 
is that you cannot easily uninstall a CMake package, for example using: 
\begin{script}
\fprompt{~/program} \gtt{sudo} \bftt{make} \rtt{uninstall}
\end{script}
\\[-0.5cm]
In the CMake case you will need to remove files manually. \\[0.25cm]
However this does not really matter when the project is prepared to be distributed by the Linux software repositories in the RPM or the DEB format. 

