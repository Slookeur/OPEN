\section{cmake}



\subsection{The "\texttt{CMakeLists.txt}" file} 

The "\texttt{CMakeLists.txt}" is a configuration file that defines:  
\begin{itemize}
\item The rules to configure your project
\item The rules to build your project
\item The rules to install your project
\end{itemize}
To start working your CMake package you will need first to create a directory to work in:
\begin{script}
\fprompt{~} mkdir program
\fprompt{~} cd program
\end{script}
Then simply use a text editor to work on this file and insert instructions regarding configuration, building and installation of your program. 

\subsubsection*{CMake version information}

The "\texttt{CMakeLists.txt}" file always starts by the minimum CMake version to use: 
\begin{script}
\confa{cmake_minimum_required} (\bftt{VERSION} 3.10)
\end{script}

\subsubsection*{Project name, version and programmation languages}

Describe project information using the \bftt{project} command:
\begin{script}
\confa{project} (prog \bftt{VERSION} 1.2.12)
\confa{project} (prog \bftt{DESCRIPTION} "A tool box")
\confa{project} (prog \bftt{HOMEPAGE_URL} "https://www.program.com")
\confa{project} (prog \bftt{LANGUAGES} C Fortran)
\end{script}
CMake will decompose the \bftt{VERSION} command in:
\begin{center}\texttt{MAJOR\_VERSION.MINOR\_VERSION.PATCH\_VERSION} \end{center}
The \bftt{LANGUAGES} option allows to check for compilers, however other dependencies might be needed (see bellow). \\[0.25cm]
For more information see the CMake \bftt{project} command related documentation \href{https://cmake.org/cmake/help/latest/command/project.html\#command:project}{here}.

\subsubsection*{Dependencies}
\label{cmake_deps}

Check for package using the \bftt{find\_package} and \bftt{pkg\_check\_modules} commands: 
\begin{script}
\confa{find_package} (PkgConfig \bftt{REQUIRED})

\comm{Checking for gtk+3.0:}
\confa{pkg_check_modules} (GTK3 \bftt{REQUIRED} IMPORTED_TARGET gtk3)

\comm{Checking for libxml-2.0:}
\confa{pkg_check_modules} (LIBXML2 \bftt{REQUIRED} IMPORTED_TARGET libxml-2.0)

\comm{Checking for libavcodec, and other FFMPEG based libraries:}
\confa{pkg_check_modules} (LIBAVUTIL \bftt{REQUIRED} IMPORTED_TARGET libavutil)
\confa{pkg_check_modules} (LIBAVCODEC \bftt{REQUIRED} IMPORTED_TARGET libavcodec)
\confa{pkg_check_modules} (LIBAVFORMAT \bftt{REQUIRED} IMPORTED_TARGET libavformat)
\confa{pkg_check_modules} (LIBSWSCALE \bftt{REQUIRED} IMPORTED_TARGET libswscale)

\comm{Checking for epoxy:}
\confa{pkg_check_modules} (EPOXY \bftt{REQUIRED} IMPORTED_TARGET epoxy)
\end{script}

\subsubsection*{Compiler options}

\begin{script}
\confa{set} (\confa{CMAKE_C_FLAGS} "-fopenmp")
\confa{set} (\confa{CMAKE_Fortran_FLAGS} "-fopenmp")
\end{script}


\subsection{Configuration using CMake}

\begin{script}
\fprompt{~/program} cmake .
\end{script}

\subsection{Building using CMake}

\begin{script}
\fprompt{~/program} cmake --build . -j 12
\end{script}

\subsection{Packaging with CPack}

\href{https://cmake.org/cmake/help/book/mastering-cmake/chapter/Packaging\%20With\%20CPack.html}{CPack} is a powerful, easy to use, 
cross-platform software packaging tool distributed with CMake. 
It uses the generators concept from CMake to abstract package generation on specific platforms. 
It can be used with or without CMake, but it may depend on some software being installed on the system. 
Using a simple configuration file, or using a CMake module, the author of a project can package a complex project into a simple installer. 
