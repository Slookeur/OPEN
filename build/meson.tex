\section{Meson}
\href{https://mesonbuild.com}{Meson} is an open source build system meant to be both fast, and as user friendly as possible. \\
Most of the information bellow is adapted from the main documentation:
\begin{center}\href{https://mesonbuild.com/Manual.html}{https://mesonbuild.com/Manual.html}\end{center}
As for the other build systems, \href{https://en.wikipedia.org/wiki/Pkg-config}{pkg-config} will be used extensively thereafter, but in the case of meson it is not mandatory to call it directly. \\
Before going further into this guide, and if you want to try to build and package your program using meson, you will have to install the following tools: 
\begin{itemize}
\item For Windows and OSX please refer to the corresponding websites.
\item For Linux you can quite conveniently use the command line as follow:
\begin{itemize}
\item Red Hat based Linux (using the \bftt{dnf} command):
\begin{scriptii}
\fprompt{~} sudo \bftt{dnf} install meson
\fprompt{~} sudo \bftt{dnf} install pkg-config gcc gcc-gfortran
\end{scriptii}
\item Debian based Linux (using the \bftt{apt} command):
\begin{scriptii}
\uprompt{~} sudo \bftt{apt} install meson
\uprompt{~} sudo \bftt{apt} install pkg-config gcc gfortran
\end{scriptii}
\end{itemize}
\end{itemize}
To learn more about software installation on Linux check out my \href{https://www.ipcms.fr/wp-content/uploads/2021/11/linux.pdf}{Linux tutorial}. \\
To learn more about scripting and the usage of the command line check out my \href{https://www.ipcms.fr/wp-content/uploads/2021/05/bash.pdf}{Bash tutorial}. 

\subsection{Example project}

Same as for the GNU autotools and CMake, see section~\ref{example_project} for more information. 

\subsection{The file: \bftt{meson.build}}

The file \bftt{meson.build} is a configuration file that defines: 
\begin{itemize}
\item The rules to configure your project
\item The rules to build your project
\item The rules to install your project
\end{itemize}
In the next pages I will browse briefly the different parts of the file \bftt{meson.build}, however my approach will remain simple. 
For detailed information check the \href{https://mesonbuild.com/Manual.html}{official user manual} \\[0.25cm]
To start working on your meson package you will need first to create a directory to work in:
\begin{script}
\fprompt{~} mkdir program
\fprompt{~} cd program
\end{script}
\\[-0.25cm]
\noindent Then simply use a text editor to work on this file and insert instructions regarding configuration, building and installation of your program:
\begin{script}
\fprompt{~/program} \bftt{vi} \rtt{meson.build}
\end{script}
\\[-0.75cm]
\noindent Note that in the \bftt{meson.build} file the following rules apply:
\begin{itemize}
\item Text strings must delimited by single quotes: \bftt{\textquotesingle}{\em{text here}}\bftt{\textquotesingle}
\item Double quotes \bftt{"} are not allowed
\end{itemize}

\subsubsection*{Project description, version and programming languages}
\label{meson_proj}
Describe project information using the \confb{project}{} instruction:
{\footnotesize{
\begin{script}
\confb{project}{\gtt{\mstr{\gtt{prog}}}, \mstr{c}, \mstr{fortran}, license : \mstr{AGPL-3.0-or-later}, version : \mstr{1.2.12}}
\end{script} 
}}
\\[-0.5cm]
\noindent Where \gtt{prog} is the project name, followed by programming languages, and other coma separated elements. \\[0.25cm]
Project information must be defined using that command, and there is only one single \confb{project}{} instruction that can be found in the file \bftt{meson.build}. \\[0.25cm]
It is possible to split the instruction on several lines for clarity purposes: 
{\footnotesize{
\begin{script}
\confb{project}{\gtt{\mstr{\gtt{prog}}}, \mstr{c}, \mstr{fortran}, 
                   license : \mstr{AGPL-3.0-or-later}, 
                   version : \mstr{1.2.12}}
\end{script} 
}}
\\[-0.5cm]
\noindent From that single instruction meson prepare several built-in variables stored in the meson \href{https://mesonbuild.com/Reference-manual\_functions.html#project}{project data structure}. 
\newpage
\noindent These variables can be accessed using the following syntax:
\begin{center} \bftt{this\_data = }\dctt{meson.}{\em{data\_name}}\texttt{()}\end{center}
\noindent Where the variable \bftt{this\_data} will be used to store the information obtained from the meson project data structure when searching for {\em{data\_name}}. \\[0.25cm]
Examples: 
\vspace{-0.25cm}
{\footnotesize{
\begin{script}
\comm{Define a variable to store the project name (prog):}
\bftt{project\_name = }\dctt{meson.}project\_name()

\comm{Define a variable to store the project license (AGPL-3.0-or-later):}
\bftt{project\_license = }\dctt{meson.}project\_license()

\comm{Define an array variable to store each part of the version information}
\comm{the .split() instruction defines how parts are being separated, in this example by dots}
\bftt{project\_version = }\dctt{meson.}project\_version().\confb{split}{\mstr{.}}
\comm{project\_version[0] = 1}
\comm{project\_version[1] = 2}
\comm{project\_version[2] = 12}
\end{script} 
}}
\\[-0.25cm]
\noindent Note that the file \bftt{meson.build} contain several others data structures for which information can be accessed similarly: 
\begin{center}\bftt{this\_data = }\dctt{data\_structure.}{\em{data\_name}}\texttt{()}\end{center}
It is also to retrieve the value of several built-in, or user defined \href{https://mesonbuild.com/Build-options.html}{options} using the \href{https://mesonbuild.com/Reference-manual\_functions.html#get\_option}{get\_option} function, the syntax is the following: 
\begin{center}\bftt{this\_option = }\confb{get\_option}{\texttt{\mstr{option\_name}}}\end{center}
\noindent Examples:
\vspace{-0.25cm}
{\footnotesize{
\begin{script}
\comm{The complete installation prefix is stored in a built-in variable named 'prefix':}
\bftt{project\_prefix = }\confb{get\_option}{\mstr{prefix}}

\comm{More illustrations page~\pageref{meson_user_options}}
\end{script} 
}}
\\[-0.5cm]
\noindent To know more about meson variables and syntax:
\begin{center}\href{https://mesonbuild.com/Syntax.html}{https://mesonbuild.com/Syntax.html}\end{center}
\noindent To display the content of a variable in the configuration output, use the \confb{message}{} instruction with coma separated elements:
\\[-0.75cm]
\begin{script}
\confb{message}{\mstr{My project name is}, project\_name}
\end{script}
\\[-0.75cm]
\noindent To obtain in the output:\\[-0.75cm]
\begin{script}
\bftt{Message:} My project name is prog
\end{script}
\newpage
\subsubsection*{Dependencies}
\label{meson_deps}

In this manual I will only focus on the \bftt{pkg-config} way of doing things, other possibilities are available and documented in the official meson documentation: 
\begin{center}\href{https://mesonbuild.com/Dependencies.html}{https://mesonbuild.com/Dependencies.html}\end{center}
Check for package(s) using the \confb{dependency}{} instruction, note that meson allows to directly use the \bftt{pkg-config} keyword as argument of this instruction:
\begin{script}
\comm{Checking for libxml-2.0:}
\bftt{xml =} \confb{dependency}{\mstr{libxml-2.0}}

\comm{Checking for libavcodec, and other FFMPEG based libraries:}
\bftt{avutil =} \confb{dependency}{\mstr{libavutil}}
\bftt{avcodec =} \confb{dependency}{\mstr{libavcodec}}
\bftt{avformat =} \confb{dependency}{\mstr{libavformat}}
\bftt{swscale =} \confb{dependency}{\mstr{libswscale}}

\comm{Declaration of a variable to store all FFMPEG dependencies:}
\confm{ffmpeg}{avutil, avcodec, avformat, swscale}

\comm{Checking for epoxy:}
\bftt{epoxy =} \confb{dependency}{\mstr{epoxy}}

\comm{Declaration of a variable to store, and easily re-use, all dependencies:}
\confm{all\_deps}{gtk, xml, glu, epoxy, ffmpeg}
\end{script}
\\[-0.25cm]
\noindent This allows to define variables, eg. \bftt{gtk}, to store the information regarding each dependencies, using the keyword in pink, eg. \magenta{\texttt{gtk+-3.0}} associated to the \bftt{pkg-config} libraries (see Sec.~\ref{clibs}). 
Note that keywords are considered as strings and must therefore be inserted between single quotes, eg. \texttt{\mstr{gtk+-3.0}}. \\
You will also need to link the project to the required libraries, but only at the moment the build target is declared, see page~\pageref{build_rules_meson}. \\[0.25cm]
By default dependencies are required, if one is not found the configuration will fail and stop. \\
However it is possible to make a dependency optional setting the \bftt{required} option to \texttt{false}, (the default value being \texttt{true}): 
\begin{script}
omp = \confb{dependency}{\mstr{openmp}, \bftt{required : }false}
\end{script}

\newpage
\subsubsection*{Compiler options}

Default compiler flags and options are automatically defined by meson at the configure stage
depending on the compiler(s) selected or available in the system. \\ 
To define additional custom compiler options you can either specify flags on the meson configure command line (see Sec.~\ref{mesonconfigure}), or, create a dedicated section in the file \bftt{meson.build}: 
\begin{script}
\comm{The type of build is stored in the meson built-in variable 'buildtype'}
build\_type = \confb{get\_option}{\mstr{buildtype}}

\comm{Then creating variables to store adapted compiler options}
\bbtt{if} build\_type == \mstr{release}
\comm{Release version}
  \confm{cc\_args}{\mstr{-O3}, \mstr{-g}}
  \confm{fc\_args}{\mstr{-O3}, \mstr{-g}}
\bbtt{elif} build\_type == \mstr{debug}
\comm{Debug version}
  \confm{cc\_args}{\mstr{-O0}, \mstr{-g3}, \mstr{-fvar-tracking}, \mstr{-fbounds-check}}
  \confm{fc\_args}{\mstr{-O0}, \mstr{-g3}, \mstr{-fvar-tracking}, \mstr{-fbounds-check}}
\bbtt{else}
\comm{Any other type of build}
  \bftt{cc\_args =}
  \bftt{fc\_args =}
\bbtt{endif}
\end{script}
\\[-0.5cm]
\noindent The variables created at this stage to store the compiler options, \bftt{cc\_args} and \bftt{fc\_args} for respectively the C and the Fortran compiler, can be used later in the build process, see page~\pageref{build_rules_meson}. 
\newpage
\subsubsection*{Preprocessor variable definitions}

To define compiler variables use the \confb{add\_project\_arguments}{} instruction:
{\texttt{
\begin{center} \confb{add\_project\_arguments}{FLAG, \bftt{language :}\ \mstr{target\_language}}\end{center}
}}
\noindent Where \texttt{FLAG} is the flag to pass to the compiler, and \magenta{\texttt{\mstr{target\_language}}} belong to the list of programming language used in the project and defined in the \confb{project}{} instruction (see Sec~\ref{meson_proj}), example:
{\footnotesize{
\begin{script}
\comm{meson handles the creation of file path easily using the / symbol:}
package\_prefix = \confb{get\_option}{\mstr{prefix}} / \confb{get\_option}{\mstr{datadir}} / \mstr{prog}

\comm{building strings is a little bit more tricky, and requires to use the .join function:}
\comm{new\_string = \textquotesingle{-}link-\textquotesingle.join(\textquotesingle{bla}\textquotesingle, \textquotesingle{bla}\textquotesingle, \textquotesingle{bla}\textquotesingle)}
\comm{ -  \textquotesingle{-}link-\textquotesingle the concatenation motif, can be empty}
\comm{ - .join(\textquotesingle{bla}\textquotesingle, \textquotesingle{bla}\textquotesingle, \textquotesingle{bla}\textquotesingle) : coma separated list of strings to concatenate}
\comm{Result, store in new\_string: \textquotesingle{bla-link-bla-link-bla}\textquotesingle}
package\_logo = \mstr{}\confb{.join}{\mstr{-DLOGO="}, package\_prefix, \mstr{"}}

\confb{add\_project\_arguments}{package\_logo, \bftt{language :} \mstr{c}}
\end{script}
}}\\[-0.5cm]
\noindent For the GNU C compiler command line, the previous instructions would be translated in:
\vspace{-0.25cm}
\begin{script}
\bftt{gcc} \dgtt{-D}LOGO=\magenta{"/usr/local/share/prog/pixmaps/logo.png"}
\end{script}
\\[-0.75cm]
\noindent Example of applications:
\begin{itemize}
\item If OpenMP is available we append the OpenMP information to the dependencies, 
and we create new pre-processor flags to activate OpenMP code in both C and Fortran:
\vspace{-0.25cm}
{\small{
\begin{scripti}
omp = \confb{dependency}{\mstr{openmp}, \bftt{required : }false}
\bbtt{if} omp.found ()
  \confm{all\_deps}{all\_deps, omp}
  \confb{add\_project\_arguments}{\mstr{-DOPENMP}, \bftt{language : }\mstr{c}}
  \confb{add\_project\_arguments}{\mstr{-DOPENMP}, \bftt{language : }\mstr{fortran}}
\bbtt{else}
  \confb{message}{\mstr{OpenMP not found}}
\bbtt{endif}
\end{scripti}
}}
\vspace{-0.75cm}
\item Using the the built-int \href{https://mesonbuild.com/Reference-manual\_builtin\_host\_machine.html#host-machine-information-host\_machine-extends-build\_machine}{\bftt{host\_machine}} data structure we check the OS, then create flag: 
\vspace{-0.75cm}
{\small{
\begin{scripti}
system = \confa{host\_machine}.system()
\bbtt{if} system == \mstr{linux}
  \confb{add\_project\_arguments}{\mstr{-DLINUX}, \bftt{language : }\mstr{c}}
\bbtt{elif} system == \mstr{windows}
  \confb{add\_project\_arguments}{\mstr{-DWINDOWS}, \bftt{language : }\mstr{c}}
\bbtt{elif} system == \mstr{darwin}
  \confb{add\_project\_arguments}{\mstr{-DOSX}, \bftt{language : }\mstr{c}}
\bbtt{endif}
\end{scripti}
}}
\end{itemize}

\subsubsection*{User defined configuration options}
\label{meson_user_options}

To define custom options for the configuration stage requires to use an external file: 
\begin{itemize}
\item For meson $<$ 1.1.10, the file name must be: \bftt{meson\_options.txt}
\item For meson $\ge$ 1.1.10, the file name must be: \bftt{meson.options}
\end{itemize}
To define an option use the \confb{option}{} command, and follow the syntax:\\[0.25cm]
{\small{
\confb{option}{\texttt{\mstr{name},~\bftt{type:}~\mstr{type},~\bftt{value:}~default\_value,~\bftt{description:}~\mstr{Blablabla}}}\\[0.5cm]
}}
Examples:
{\footnotesize{
\begin{script}
\confb{option}{\mstr{gtk}, \bftt{type:} \mstr{integer}, \bftt{value:} 3, \bftt{description:} \mstr{Version of the GTK library}}
\confb{option}{\mstr{openmp}, \bftt{type:} \mstr{boolean}, \bftt{value:} true, \bftt{description:} \mstr{Use OpenMP}}
\end{script}
}}
\\[-0.5cm]
\noindent More information is available on the \href{https://mesonbuild.com/Build-options.html}{build options} page of the user manual. \\[0.25cm] 
Options declared in this file can then be used in the meson configuration line, and in the file \bftt{meson.build}. \\
In the file \bftt{meson.build} it is required to retrieve the value of the option using the \confb{get\_option}{} instruction:
\begin{itemize}
\item To trigger build with different versions of the GTK library:
{\footnotesize{
\begin{scripti}
gtk\_version = \confb{get\_option}{\mstr{gtk}}
\bbtt{if} gtk\_version == 3
  \comm{Default option: checking for gtk+3.0:}
  \bftt{gtk =} \confb{dependency}{\mstr{gtk+-3.0}}
  \confb{add\_project\_arguments}{\mstr{-DGTK3}, \bftt{language : }\mstr{c}}
\bbtt{else}
  \comm{Using -Dgtk=4 on the meson configuration command line}
  \bftt{gtk =} \confb{dependency}{\mstr{gtk4}}
  \confb{add\_project\_arguments}{\mstr{-DGTK4}, \bftt{language : }\mstr{c}}
\bbtt{endif}
\end{scripti}
}}
\item To check if OpenMP is available and adjust the compilation options accordingly:
{\footnotesize{
\begin{scripti}
use\_openmp = \confb{get\_option}{\mstr{openmp}}
\bbtt{if} use\_openmp
  \bftt{omp =} \confb{dependency}{\mstr{openmp}, \bftt{required : }false}
  \bftt{if} omp.found ()
    \confm{all\_deps}{all\_deps, omp}
    \confb{add\_project\_arguments}{\mstr{-DOPENMP}, \bftt{language : }\mstr{c}}
    \confb{add\_project\_arguments}{\mstr{-DOPENMP}, \bftt{language : }\mstr{fortran}}
  \bbtt{else}
    \confb{message}{\mstr{OpenMP not found: building serial version}}
  \bbtt{endif}
\bbtt{endif}
\end{scripti}
}}
\end{itemize}
\vspace{-0.5cm}

\subsubsection*{Declaring the project sources}

In meson you declare sources in list of file(s), that your refer to using a variable name. 
This variable will contain the list of all files to be used to build the program. 
Furthermore in meson it is not possible to search for file(s) recursively, therefore all files to be used must be declared in the file \bftt{meson.build}:
\begin{script}
\confm{src\_c}{\mstr{src/main.c}, \mstr{src/gui.c}} 
\confm{src\_f}{\mstr{src/file-1.f90}, \mstr{src/file-2.f90}}
\confm{src\_all}{src\_c, src\_f}
\end{script}
\\[-0.5cm]
\noindent \bftt{src\_c} is the name of a variable that will contain the list of files in C. \\[0.25cm]
\bftt{src\_f} is the name of a variable that will contain the list of files in Fortran. \\[0.25cm]
\bftt{src\_all} constructed using the previous variables is the list of all source files. \\[0.25cm]
Note that the complete list of source file(s) must be provided in the file \bftt{meson.build}. 

\subsubsection*{Declaring project headers include directories}

In a meson project it is required to declare project headers (\texttt{".h"}) include directory(ies) before building instructions. 
This is done using the \confb{include\_directories}{} instruction followed by a list of coma separated file paths. \\
Save the result of these command in a variable that will be called when building the project:
\vspace{-0.25cm}
\begin{script}
\bftt{inc\_dir =} \confb{include\_directories}{\mstr{.}, \mstr{src}}
\end{script}
\\[-0.75cm]
Equivalent to:
\vspace{-0.125cm}
\begin{script}
\confm{all\_inc}{\mstr{.}, \mstr{src}}
\bftt{inc\_dir =} \confb{include\_directories}{all\_inc}
\end{script}
\\[-0.25cm]
\noindent \bftt{inc\_dir} contains the list of all file paths to search for include files. 
\newpage

\subsubsection*{Declaring the project building process}
\label{build_rules_meson}

In meson you declare the project building process using the \confb{executable}{} instruction. \\
The \confb{executable}{} uses all information prepared up to this point: sources, include directories, dependencies, flags ... 
\begin{script}
\confb{executable}{\bftt{project\_name}, 
             \bftt{sources :} src\_all, 
             \bftt{include\_directories :} inc\_dir, 
             \bftt{dependencies :} all\_deps, 
             \bftt{c\_args : } cc\_args, 
             \bftt{fortran\_args: } fc\_args,
             \bftt{install :} true}
\end{script}
\\[-0.5cm]
\noindent The first argument of the command is the name of the binary to build, in this case the name of the project.  
It is followed by a coma separated list of elements, to provide information on the build process, by order of appearance in this example:
\begin{itemize}
\item The sources, using the keyword \bftt{source}
\item The include directory(ies), using the keyword \bftt{include\_directories} 
\item The dependencies, using the keyword \bftt{dependencies}
\item User defined C compiler flags, using the keyword \bftt{c\_args}
\item User defined Fortran compiler flags, using the keyword \bftt{fortran\_args}
\item The installation option, using the keyword \bftt{install}
\end{itemize}
The syntax is similar in each case:
\begin{center}\bftt{keyword : }\em{value}\end{center}
\noindent Note that variables created in the previous sections of the file \bftt{meson.build} are used to provide values for each field.  \\
The last keyword in the \confb{executable}{} instruction, tells if the binary is to be installed or not:
\begin{center}\bftt{install : }{\em{true}} \texttt{/} {\em{false}}\end{center}
\noindent It is possible to find multiple \confb{executable}{} instructions in the file \bftt{meson.build}. \\[0.25cm]
Note that objects built using the \confb{executable}{} instruction are automatically installed in "\texttt{\confb{get\_option}{\mstr{prefix}}/bin}" if you need to install it somewhere else then add the "\bftt{install\_dir~:~}{\em{installation\_path}}" keyword to the \confb{executable}{} instruction:
%\vspace{-0.25cm}
\begin{script}
\bftt{package\_libexec =} \confb{get\_option}{\mstr{libexecdir}}
\confb{executable}{\mstr{alternate\_bin}, \bftt{sources :} alt\_src, 
             \bftt{include\_directories :} alt\_inc, \bftt{dependencies :} alt\_deps,
             \bftt{install :} true, \bftt{install\_dir :} package\_libexec}
\end{script}
\newpage
\subsubsection*{Installation using meson}

Binary (or binaries) built using the file \bftt{meson.built} are automatically installed provided the "\bftt{install~:}~\texttt{true}" information appear in the \confb{executable}{} instruction. \\[0.25cm]
To generate installation rules for additional data for your project use the \confb{install\_<FUNC>}{} instructions:
\begin{itemize}
\item To install file, use the \confb{install\_data}{} instruction:
\begin{scripti}
\confb{install\_dat}{\mstr{metadata/program-mime.xml}, 
              \bftt{install\_dir : }data\_dir / \mstr{mime/packages}}
\confb{install\_dat}{\mstr{metadata/com.program.www.appdata.xml},
              \bftt{install\_dir : }data\_dir / \mstr{metainfo}}
\confb{install\_dat}{\mstr{metadata/program.desktop},
               \bftt{install\_dir : }data\_dir / \mstr{applications}}
\confb{install\_dat}{\mstr{ChangeLog}, \bftt{install\_dir : }data\_dir / \mstr{doc/prog}}
\confb{install\_dat}{\mstr{COPYING}, \bftt{install\_dir : }data\_dir / \mstr{doc/prog}}
\end{scripti}
\item To install a directory recursively, use the \confb{install\_subdir}{} instruction:
\begin{scripti}
\confb{install\_subdir}{\mstr{data}, \bftt{install\_dir : }data\_dir / \mstr{prog}}
\confb{install\_subdir}{\mstr{pixmaps}, \bftt{install\_dir : }data\_dir / \mstr{prog}}
\end{scripti}
\end{itemize}

\subsubsection*{Post-installation script}

To use apply post-installation configuration, including system integration, an option is to use a
script call by the \confa{meson}.\confb{add\_install\_script}{} instruction:
\begin{script}
\comm{Post installation using a script, syntax:}
\comm{meson.add\_install\_script(script\_name, agr\_1, arg\_2 ...)}
\comm{arguments for the script follow its name, and are separated by coma:}
\confa{meson}.\confb{add\_install\_script}{\mstr{post\_install.sh}, \bftt{project\_prefix}}
\end{script}
This will execute the command \texttt{./post-install.sh} which is a script containing the post-installation instructions. 
It is required to transmit to the script the installation directory selected by meson, this is done using the variable \bftt{project\_prefix} in argument of the script. 
An example of content for the script \texttt{post-install.sh} is provided in appendix~\ref{poscript}. \\[0.25cm]
A complete example of the file \bftt{meson.build} is provided in appendix~\ref{mesonbuild}.

\subsection{Configuring, building and installing a meson package}

To configure, build and install a meson package use the \bftt{meson} command. \\
In the following I will consider that the file \bftt{meson.build} was completed: 
{\footnotesize{
\begin{script}
\fprompt{~/program} ls -lh
-rw-r--r--. 1 user group 2,8K 24 mars  12:25 ChangeLog
-rw-r--r--. 1 user group 5,3K 24 mars  12:25 \rtt{meson.build}
-rw-r--r--. 1 user group 0,3K 24 mars  12:25 \rtt{meson.options} \comm{Optional}
-rw-r--r--. 1 user group  34K 24 mars  12:25 COPYING
drwxr-xr-x. 2 user group 4,0K 24 mars  12:25 \btt{data}
drwxr-xr-x. 2 user group 4,0K 24 mars  12:25 \btt{metadata}
drwxr-xr-x. 4 user group 4,0K 24 mars  12:25 \btt{pixmaps}
-rw-r--r--. 1 user group 4,8K 24 mars  12:25 README
drwxr-xr-x. 2 user group 4,0K 24 mars  12:25 \btt{src}
\end{script}
}}

\subsubsection*{Configuration}
\label{mesonconfigure}

To configure your meson project use the \bftt{meson} command: 
\begin{itemize}
\item With the \rtt{setup} argument
\item Followed by the name of a directory to be created
\end{itemize}
\vspace{-1cm}
\begin{script}
\fprompt{~/program} \bftt{meson} \rtt{setup} \dctt{buildir}
\end{script}
\\[-0.5cm]
\noindent The last argument of the command is the new directory where the program will be built, \\
in this example case \dctt{buildir}. \\
The result of this command is both: 
\begin{itemize}
\item The creation of several files and directories in \dctt{buildir}:
\begin{scripti}
\fprompt{~/program} ls buildir
\bftt{prog.p}      compile\_commands.json  \btt{ meson-logs}
build.ninja  \btt{meson-info}             \btt{meson-private}
\end{scripti}
\item The configuration of your project that is already ready to be built using default options.
\end{itemize}
Other options can be added to the command line using the \texttt{-Doption} syntax, example:
\begin{script}
\fprompt{~/program} \bftt{meson} \rtt{setup} \dctt{buildir} -Dgtk=4
\end{script}
\noindent Some options might be built-in but configuring with user defined options requires to create and prepare the file \bftt{meson.options} accordingly. \\
To list available options use: 
\begin{script}
\fprompt{~/program} \bftt{meson} \rtt{configure} \dctt{buildir}
\end{script}

\subsubsection*{Building}

To build use the \bftt{meson} command with the \rtt{compile} argument: 
\begin{itemize}
\item Followed by the option \gtt{-C} and the name of the build directory:
\begin{scripti}
\fprompt{~/program} \bftt{meson} \rtt{compile} \gtt{-C} \dctt{buildir}
\end{scripti}
\item Without option inside the build directory:
\begin{scripti}
\fprompt{~/program} cd \dctt{buildir}
\fprompt{~/program/buildir} \bftt{meson} \rtt{compile}
\end{scripti}
\end{itemize}

\subsubsection*{Installing}

To build use the \bftt{meson} command with the \rtt{install} argument: 
\begin{itemize}
\item Followed by the option \gtt{-C} and the name of the build directory:
\begin{scripti}
\fprompt{~/program} \gtt{sudo} \bftt{meson} \rtt{install} \gtt{-C} \dctt{buildir}
\end{scripti}
\item Without option inside the build directory:
\begin{scripti}
\fprompt{~/program} cd \dctt{buildir}
\fprompt{~/program/buildir} \gtt{sudo} \bftt{meson} \rtt{install}
\end{scripti}
\end{itemize}


\subsection{Comparison with CMake and the GNU autotools}

As for CMake it is not possible to easily uninstall a meson package, for example using: 
\begin{script}
\fprompt{~/program} \gtt{sudo} \bftt{meson} \rtt{uninstall}
\end{script}
\\[-0.5cm]
In the meson case you will need to remove files manually. \\[0.25cm]
However this does not really matter when the project is prepared to be distributed by the Linux software repositories in the RPM or the DEB format. 

