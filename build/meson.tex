\section{Meson}
\begin{center} \red{\Huge{<-{-} Work In Progress {-}->}} \end{center}
\href{https://mesonbuild.com}{Meson} is an open source build system meant to be both extremely fast, and, even more importantly, as user friendly as possible. \\
Most of the information bellow is adapted from the main documentation:
\begin{center}\href{https://mesonbuild.com/Manual.html}{https://mesonbuild.com/Manual.html}\end{center}
As for the other build systems, \href{https://en.wikipedia.org/wiki/Pkg-config}{pkg-config} will be used extensively thereafter, but in the case of meson it is not mandatory to call it directly. \\
Before going further into this guide, and if you want to try to build and package your pogram using meson, you will have to install the following tools: 
\begin{itemize}
\item For Windows and OSX please refer to the corresponding websites.
\item For Linux you can quite conveniently use the command line as follow:
\begin{itemize}
\item Red Hat based Linux (using the \bftt{dnf} command):
\begin{scriptii}
\fprompt{~} sudo \bftt{dnf} install meson
\fprompt{~} sudo \bftt{dnf} install pkg-config gcc gcc-gfortran
\end{scriptii}
\item Debian based Linux (using the \bftt{apt} command):
\begin{scriptii}
\uprompt{~} sudo \bftt{apt} install meson
\uprompt{~} sudo \bftt{apt} install pkg-config gcc gfortran
\end{scriptii}
\end{itemize}
\end{itemize}
To learn more about software installation on Linux check out my \href{https://www.ipcms.fr/wp-content/uploads/2021/11/linux.pdf}{Linux tutorial}. \\
To learn more about scripting and the usage of the command line check out my \href{https://www.ipcms.fr/wp-content/uploads/2021/05/bash.pdf}{Bash tutorial}. 

\subsection{Example project}

Same as for the GNU autotools and CMake, see section~\ref{example_project} for more information. 

\subsection{The file: \bftt{meson.build}}

The file \bftt{meson.build} is a configuration file that defines: 
\begin{itemize}
\item The rules to configure your project
\item The rules to build your project
\item The rules to install your project
\end{itemize}
In the next pages I will browse briefly the differents parts of the file \bftt{meson.build}, however my approach will remain simple. 
For detailed information check the \href{https://mesonbuild.com/Manual.html}{official user manual} \\[0.25cm]
To start working on your meson package you will need first to create a directory to work in:
\begin{script}
\fprompt{~} mkdir program
\fprompt{~} cd program
\end{script}
\\[-0.25cm]
\noindent Then simply use a text editor to work on this file and insert instructions regarding configuration, building and installation of your program:
\begin{script}
\fprompt{~/program} \bftt{vi} \rtt{meson.build}
\end{script}
\\[-0.75cm]
\noindent Note that in the \bftt{meson.build} file the following rules apply:
\begin{itemize}
\item Text strings must delimited by single quotes: \bftt{\textquotesingle}{\em{text here}}\bftt{\textquotesingle}
\item Double quotes \bftt{"} are not allowed
\end{itemize}

\subsubsection*{Project description, version and programming languages}
\label{meson_proj}
Describe project information using the \confb{project}{} instruction:
{\footnotesize{
\begin{script}
\confb{project}{\gtt{\mstr{\gtt{prog}}}, \mstr{c}, \mstr{fortran}, license : \mstr{AGPL-3.0-or-later}, version : \mstr{1.2.12}}
\end{script} 
}}
\\[-0.5cm]
\noindent Where \gtt{prog} is the project name, followed by programming languages, and other coma separated elements. \\[0.25cm]
Project information must be defined using that command, and there is only one single \confb{project}{} instruction that can be found in the file \bftt{meson.build}. \\[0.25cm]
It is possible to split the instruction on several lines clarity purposes. \\[0.25cm]
From that single instruction meson prepare several built-in variables stored in the meson \href{https://mesonbuild.com/Reference-manual\_functions.html#project}{project data structure}. 
These variables can be accessed using the following syntax:
{\footnotesize{
\begin{script}
\comm{Define a variable to store the project name (prog):}
project\_name = \dctt{meson.}project\_name()

\comm{Define a variable to store the project license (AGPL-3.0-or-later):}
project\_license = \dctt{meson.}project\_license()

\comm{Define an array variable to store each part of the version information}
\comm{the .split() instruction defines how parts are being separated, in this example by dots:}
project\_version = \dctt{meson.}project\_version().\confb{split}{\mstr{.}}
\comm{project\_version[0] = 1}
\comm{project\_version[1] = 2}
\comm{project\_version[2] = 12}
\end{script} 
}}
\\[-0.25cm]
\noindent To know more about meson variables and syntax:
\begin{center}\href{https://mesonbuild.com/Syntax.html}{https://mesonbuild.com/Syntax.html}\end{center}
\noindent To display the content of a variable in the configuration output, use the \confb{message}{} instruction with coma separated elements:
\begin{script}
\confb{message}{\mstr{My project name is}, project\_name}
\end{script}
\\[-0.75cm]
\noindent To obtain in the output:
\begin{script}
\bftt{Message:} My project name is prog
\end{script}
\\[-1.5cm]

\subsubsection*{Dependencies}
\label{meson_deps}

In this manual I will only focus on the \bftt{pkg-config} way of doing things, other possibilites are available and documented in the official meson documentation: 
\begin{center}\href{https://mesonbuild.com/Dependencies.html}{https://mesonbuild.com/Dependencies.html}\end{center}
Check for package(s) using the \confb{dependency}{} instruction, note that meson allows to directly use the \bftt{pkg-config} keyword as argument of this instruction:
\begin{script}
\comm{Checking for gtk+3.0:}
\bftt{gtk =} \confb{dependency}{\mstr{gtk+-3.0}}

\comm{Checking for libxml-2.0:}
\bftt{xml =} \confb{dependency}{\mstr{libxml-2.0}}

\comm{Checking for libavcodec, and other FFMPEG based libraries:}
\bftt{avutil =} \confb{dependency}{\mstr{libavutil}}
\bftt{avcodec =} \confb{dependency}{\mstr{libavcodec}}
\bftt{avformat =} \confb{dependency}{\mstr{libavformat}}
\bftt{swscale =} \confb{dependency}{\mstr{libswscale}}

\comm{Declaration of a variable to store all FFMPEG dependencies:}
\confm{ffmpeg}{avutil, avcodec, avformat, swscale}

\comm{Checking for epoxy:}
\bftt{epoxy =} \confb{dependency}{\mstr{epoxy}}

\comm{Declaration of a variable to store, and easily re-use, all dependencies:}
\confm{all\_deps}{gtk, xml, glu, epoxy, ffmpeg}

\end{script}
\\[-0.25cm]
\noindent This allows to define variables, eg. \bftt{gtk}, to store the information regarding each depencies, using the keyword in pink, eg. \magenta{\texttt{gtk+-3.0}} associated to the \bftt{pkg-config} librairies (see Sec.~\ref{clibs}). 
Note that keywords are considered as strings and must therefore be inserted between single quotes, eg. \texttt{\mstr{gtk+-3.0}}. \\
You will also need to link the project to the required libraries, but only at the moment the build target is declared, see page~\pageref{build_rules_meson}. 


\subsubsection*{Compiler options}

\newpage
\subsubsection*{Preprocessor variable definitions}

To define compiler variables use the \confb{add\_project\_arguments}{} instruction:
{\texttt{
\begin{center} \confb{add\_project\_arguments}{FLAG, \bftt{language :}\ \mstr{target\_language}}\end{center}
}}
\noindent Where \texttt{FLAG} is the flag to pass to the compiler, and \magenta{\mstr{target\_language}} belong to the list of programming language used in the project and defined in the \confb{project}{} instruction (see Sec~\ref{meson_proj}), example:
{\footnotesize{
\begin{script}
\confb{add\_project\_arguments}{\mstr{-DOPENMP}, \bftt{language :} \mstr{c}}
\confb{add\_project\_arguments}{\mstr{-DOPENMP}, \bftt{language :} \mstr{fortran}}
\confb{add\_project\_arguments}{\mstr{-DPACKAGE\_LOGO="pixmaps/logo.png"}, \bftt{language :} \mstr{c}}
\end{script}
}}\\[-0.5cm]
\noindent For the GNU C compiler command line, the previous instructions would be translated in:
\vspace{-0.25cm}
\begin{script}
\bftt{gcc} \dgtt{-D}OPENMP \dgtt{-D}PACKAGE\_LOGO=\magenta{"pixmaps/logo.png"}
\end{script}
\\[-0.75cm]
\noindent And for the GNU Fortran compiler command line, it would be translated in:
\vspace{-0.25cm}
\begin{script}
\bftt{gfortran} \dgtt{-D}OPENMP
\end{script}
\\[-0.5cm]
\noindent Example of applications:
\begin{itemize}
\item Utilisation of OpenMP instructions: we check if OpenMP is available, 
if yes we append the OpenMP information to the list of dependencies, 
and we create new pre-processor flags to activate OpenMP instructions in both C and Fortran code:
\begin{scripti}
omp = \confb{dependency}{\mstr{openmp}, \bftt{required : }false}
\bbtt{if} omp.found ()
  \confm{all\_deps}{all\_deps, omp}
  \confb{add\_project\_arguments}{\mstr{-DOPENMP}, \bftt{language : }\mstr{c}}
  \confb{add\_project\_arguments}{\mstr{-DOPENMP}, \bftt{language : }\mstr{fortran}}
\bbtt{else}
  \confb{message}{\mstr{OpenMP not found}}
\bbtt{endif}
\end{scripti}
\item Checking for the operating system (uisng the the built-int \href{https://mesonbuild.com/Reference-manual\_builtin\_host\_machine.html#host-machine-information-host\_machine-extends-build\_machine}{\bftt{host\_machine}} data structure) and creating preprocessor flag accordingly: 
\begin{scripti}
system = \confa{host\_machine}.system()
\bbtt{if} system == \mstr{linux}
  \confb{add\_project\_arguments}{\mstr{-DLINUX}, \bftt{language : }\mstr{c}}
\bbtt{elif} system == \mstr{windows}
  \confb{add\_project\_arguments}{\mstr{-DWINDOWS}, \bftt{language : }\mstr{c}}
\bbtt{elif} system == \mstr{darwin}
  \confb{add\_project\_arguments}{\mstr{-DOSX}, \bftt{language : }\mstr{c}}
\bbtt{endif}
\end{scripti}
\end{itemize}

\subsubsection*{Declaring the project sources}

In meson you declare sources in list of file(s), that your refer to using a variable name. 
This variable will contain the list of all files to be used to build the program. 
Furthemore in meson it is not posssible to search for file(s) recursively, therefore all files to be used must be declared in the file \bftt{meson.build}:
\begin{script}
\confm{src\_c}{\mstr{src/main.c}, \mstr{src/gui.c}} 
\confm{src\_f}{\mstr{src/file-1.f90}, \mstr{src/file-2.f90}}
\confm{src\_all}{src\_c, src\_f}
\end{script}
\\[-0.5cm]
\noindent \bftt{src\_c} is the name of a variable that will contain the list of files in C. \\[0.25cm]
\bftt{src\_f} is the name of a variable that will contain the list of files in Fortran. \\[0.25cm]
\bftt{src\_all} constructed using the previous variables is the list of all source file(s). \\[0.25cm]
Note that the complete list of source file(s) must be provided in the file \bftt{meson.build}. 

\subsubsection*{Declaring project headers include directories}

In a meson project it is required to declare project headers (\texttt{".h"}) include directory(ies) before building instructions. 
This is done using the \confb{include\_directories}{} instruction followed by a list of coma separated file paths. \\
Save the result of these command in a variable that will be called when building the project: 
\begin{script}
\bftt{inc\_dir =} \confb{include\_directories}{\mstr{.}, \mstr{src}}
\end{script}
\\[-0.75cm]
Equivalent to:
\vspace{-0.125cm}
\begin{script}
\confm{all\_inc}{\mstr{.}, \mstr{src}}
\bftt{inc\_dir =} \confb{include\_directories}{all\_inc}
\end{script}
\\[-0.75cm]
\noindent \bftt{inc\_dir} contains the list of all file paths to search for include files. 

\subsubsection*{Declaring the project building process}
\label{build_rules_meson}

In meson you declare the project building process using the \confb{executable}{} instruction. \\
The \confb{executable}{} uses all information prepared up to this point: sources, include directories, dependencies. 
\begin{script}
\confb{executable}{\bftt{project\_name}, 
             \bftt{sources :} src\_all, 
             \bftt{include\_directories :} inc\_dir, 
             \bftt{dependencies :} alldeps, 
             \bftt{c\_args : } cc\_args, 
             \bftt{fortran\_args: } fc\_args,
             \bftt{install :} true}
\end{script}
\\[-0.5cm]
\noindent Coma separeted elements, by order of appearence in this example are:
\begin{itemize}
\item The name of the binary to build
\item The sources, using the keyword \bftt{source}
\item The include directory(ies), using the keyword \bftt{include\_directories} 
\item The dependencies, using the keyword \bftt{dependencies}
\item User defined C compiler flags, using the keyword \bftt{c\_args}
\item User defined Fortran compiler flags, using the keyword \bftt{fortran\_args}
\item The installation option, using the keyword \bftt{install}
\end{itemize}
The syntax is similar in each case:
\begin{center}\bftt{keyword :} \em{value}\end{center}


\begin{center} \red{\Huge{<!-{-} Work In Progress {-}-!>}} \end{center}
