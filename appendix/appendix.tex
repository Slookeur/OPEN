
\chapter{Build system files}

\clearpage

\section{The GNU Autotools files}

\subsection{The file: \bftt{configure.ac}}
\label{configall}
\vspace{-0.5cm}
{\tiny{
\begin{script}
\confb{AC\_PREREQ}{\red{2.59}}
m4\_define(\bftt{major\_version}, \red{1})
m4\_define(\bftt{minor\_version}, \red{2})
m4\_define(\bftt{patch\_version}, \red{12})
m4\_define(\bftt{version}, \bftt{major\_version}\rtt{.}\bftt{minor\_version}\rtt{.}\bftt{patch\_version})
m4\_define(\bftt{bug\_email}, \red{\email})
m4\_define(\bftt{tar\_name}, \red{program})
m4\_define(\bftt{project\_url}, \red{https://www.program.com})
\confb{AC\_INIT}{\cstr{\bftt{prog}}, \cstr{\bftt{version}}, \cstr{\bftt{bug\_email}}, \cstr{\bftt{tar\_name}}, \cstr{\bftt{project\_url}}}
\confa{AM\_INIT\_AUTOMAKE}
\cact{AC\_DEFINE}{MAJOR\_VERSION}{\bftt{major\_version}}{Program major version}
\confb{AC\_SUBST}{MAJOR\_VERSION, \bftt{major\_version}}
\cact{AC\_DEFINE}{MINOR\_VERSION}{\bftt{minor\_version}}{Program minor version}
\confb{AC\_SUBST}{MINOR\_VERSION, \bftt{minor\_version}}
\cact{AC\_DEFINE}{PATCH\_VERSION}{\bftt{patch\_version}}{Program patch version}
\confb{AC\_SUBST}{PATCH\_VERSION, \bftt{patch\_version}}
\cacf{AC\_CHECK\_PROG}{PKG\_CONFIG}{pkg-config}{yes}{no}
\cacf{AC\_CHECK\_PROG}{UP\_MIME}{update-mime-database}{yes}{no}
\cacf{AC\_CHECK\_PROG}{UP\_DESKTOP}{update-desktop-database}{yes}{no}
\cacf{AC\_CHECK\_PROG}{UP\_APPSTREAM}{appstream-util}{yes}{no}

\dnl{Definig the macro AX\_CHECK\_COMPILER\_FLAG:}
\confb{AC\_DEFUN}{\cstr{AX\_CHECK\_COMPILER\_FLAG},
\tabul \cstr{\confb{AC\_PREREQ}{\rtt{2.64}} dnl for \_AC\_LANG\_PREFIX and AS\_VAR\_IF
\tabul \cacb{AS\_VAR\_PUSHDEF}{CACHEVAR}{ax\_cv\_check\_[]\_AC\_LANG\_ABBREV[]flags\_\$4\_\$1}
\tabul \confb{AC\_CACHE\_CHECK}{\cstr{whether \_AC\_LANG compiler accepts \$1}, CACHEVAR, \cstr{
\tabul \tabul ax\_check\_save\_flags=\$[]\_AC\_LANG\_PREFIX[]FLAGS
\tabul \tabul \_AC\_LANG\_PREFIX[]FLAGS="\$[]\_AC\_LANG\_PREFIX[]FLAGS \$4 \$1"
\tabul \tabul \confb{AC\_COMPILE\_IFELSE}{\cstr{m4\_default([\$5],[AC\_LANG\_PROGRAM()]},
\tabul \tabul \tabul \cstr{\confb{AS\_VAR\_SET}{CACHEVAR,\cstr{yes}}},
\tabul \tabul \tabul \cstr{\confb{AS\_VAR\_SET}{CACHEVAR,\cstr{no}}}}
\tabul \tabul \_AC\_LANG\_PREFIX[]FLAGS=\$ax\_check\_save\_flags}}
\tabul \confb{AS\_VAR\_IF}{CACHEVAR,yes,
\tabul \tabul [m4\_default([\$2], :)],
\tabul \tabul [m4\_default([\$3], :)]}
\tabul \confb{AS\_VAR\_POPDEF}{[CACHEVAR]}
}}\dnl{AX\_CHECK\_COMPILER\_FLAG}
\confa{AC\_PROG\_CC}
\confb{AX\_CHECK\_COMPILE\_FLAG}{\cstr{\red{\${CFLAGS}}}}
\confa{AC\_FC\_WRAPPERS}
\confb{AC\_LANG\_PUSH}{\cstr{Fortran}}
\confb{AC\_PROG\_FC}{\cstr{xlf95 fort ifort ifc f95 g95 pgf95 lf95 xlf90 f90 pgf90 gfortran}, 
             \cstr{90}}
\confb{AC\_FC\_SRCEXT}{f90, FCFLAGS\_f90="\${FCFLAGS\_f90} \${FCFLAGS}",
                AC\_MSG\_ERROR(\cstr{Err. comp. .f90})}
\confb{AC\_FC\_SRCEXT}{F90, FCFLAGS\_F90="\${FCFLAGS\_F90} \${FCFLAGS}", 
                AC\_MSG\_ERROR(\cstr{Err. comp. .F90})}
\confb{AX\_CHECK\_COMPILE\_FLAG}{\cstr{\red{\${FCFLAGS}}}}
\confa{AC\_FC\_LIBRARY\_LDFLAGS}
\confa{AC\_FC\_FREEFORM}
\confb{AC\_LANG\_POP}{\cstr{Fortran}}

\cacb{PKG\_CHECK\_MODULES}{LIBXML2}{\bftt{libxml-2.0 >= 2.4.0}}
\cacb{PKG\_CHECK\_MODULES}{GLU}{\bftt{glu}}
\cacb{PKG\_CHECK\_MODULES}{EPOXY}{\bftt{epoxy}}
\cacb{PKG\_CHECK\_MODULES}{FFMPEG}{\bftt{libavcodec libavformat libavutil libswscale}}
\confb{AC\_MSG\_CHECKING}{\cstr{the GTK version to use}}
\confb{AC\_ARG\_WITH}{\cstr{gtk},
  \cstr{\confb{AS\_HELP\_STRING}{[\magenta{--with-gtk=3|4}], [\magenta{the GTK version to use (default: 3)}]}},
  \cstr{case "\magenta{\$with\_gtk}" in
       \rtt{3}|\rtt{4}) ;;
       *) \confb{AC\_MSG\_ERROR}{[\magenta{invalid GTK version specified}]} ;;
         esac},
  \cstr{with\_gtk=3}}
\confb{AC\_MSG\_RESULT}{\cstr{\magenta{\$with\_gtk}}}
\cacb{AM\_CONDITIONAL}{GTK3}{test "\magenta{\$with\_gtk}" == "\magenta{3}"}
case "\magenta{\$with\_gtk}" in
  \rtt{3}) gtk\_version="\magenta{gtk+-3.0}"
     gtk\_version\_number="\magenta{3.16}"
     ;;
  \rtt{4}) gtk\_version="\magenta{gtk4}"
     gtk\_version\_number="\magenta{4.6}"
     ;;
esac
\cacb{PKG\_CHECK\_MODULES}{LGTK}{\magenta{\$gtk\_version} \bftt{>=} \magenta{\$gtk\_version\_number}}

\confb{AC\_CONFIG\_FILES}{\cstr{
  \red{Makefile
  src/Makefile}
}}

\confa{AC\_OUTPUT}
\end{script}
}}
\vspace{-1.5cm}
\subsection{The file: \bftt{Makefile.am}}
\label{makeall}
\vspace{-0.5cm}
{\tiny{
\begin{script}
\dgtt{prog\_datadir} = \var{\$(DESTDIR)\$(datadir)}
\dgtt{prog\_docdir} = \var{\$(DESTDIR)\$(docdir)}
\dgtt{prog\_mandir} = \var{\$(DESTDIR)\$(mandir)}
\dgtt{prog\_pkgdatadir} = \var{\$(DESTDIR)\$(pkgdatadir)}
\dgtt{prog\_desktopdir} = \var{\$(prog\_pkgdatadir)}/applications
\dgtt{prog\_metadir} = \var{\$(prog\_pkgdatadir)}/metainfo
\dgtt{prog\_mimedir} = \var{\$(prog\_pkgdatadir)}/mime/packages
\dgtt{prog\_iconsdir} = \var{\$(prog\_pkgdatadir)}/pixmaps
\dgtt{SUBDIRS} = src
\dgtt{prog\_docdir} = \var{\$(docdir)}
\dgtt{prog\_doc\_DATA} = README.md AUTHORS ChangeLog
\dgtt{prog\_mandir} = \var{\$(mandir)}/man1/
\dgtt{prog\_man\_DATA} = program.1.gz
\rtt{install-data-local:}
\tabul if test -d \var{\$(srcdir)}/data; then \textbackslash
\tabul \tabul \var{\$(mkinstalldirs) \$(prog\_pkgdatadir)}/data; \textbackslash
\tabul \tabul for data in \var{\$(srcdir)}/data/*; do \textbackslash
\tabul \tabul \tabul if test -f \var{\$\$data}; then \textbackslash
\tabul \tabul \tabul \tabul \var{\$(INSTALL\_DATA) \${data} \$(prog\_pkgdatadir)}/data; \textbackslash
\tabul \tabul \tabul fi \textbackslash
\tabul \tabul done \textbackslash
\tabul fi
\tabul if test -d \var{\$(srcdir)}/pickups; then \textbackslash
\tabul \tabul \var{\$(mkinstalldirs) \$(prog\_pkgdatadir)}/pixmaps; \textbackslash
\tabul \tabul for pixmap in \var{\$(srcdir)}/pixmaps/*; do \textbackslash
\tabul \tabul \tabul if test -f \var{\$\${pixmap}}; then \textbackslash
\tabul \tabul \tabul \tabul \var{\$(INSTALL\_DATA) \$\${pixmap} \$(prog\_pkgdatadir)}/pixmaps; \textbackslash
\tabul \tabul \tabul else \textbackslash
\tabul \tabul \tabul \tabul \var{\$(mkinstalldirs) \$(prog\_pkgdatadir)/\$\${pixmap}}	; \textbackslash
\tabul \tabul \tabul \tabul for pixma in \var{\$\${pixmap}}/*; do \textbackslash
\tabul \tabul \tabul \tabul \tabul if test -f \var{\$\${pixma}}; then \textbackslash
\tabul \tabul \tabul \tabul \tabul \tabul \var{\$(INSTALL\_DATA) \$\${pixma} \$(prog\_pkgdatadir)/\$\${pixmap}}; \textbackslash
\tabul \tabul \tabul \tabul \tabul fi \textbackslash
\tabul \tabul \tabul \tabul done \textbackslash
\tabul \tabul \tabul fi \textbackslash
\tabul \tabul done \textbackslash
\tabul fi
\tabul if [ ! -d \var{\$(prog\_iconsdir)} ]; then \textbackslash
\tabul \tabul \var{\$(mkinstalldirs)} \var{\$(prog\_iconsdir)}; \textbackslash
\tabul fi
\tabul \var{\$(INSTALL\_DATA) \$(srcdir)}/metadata/icons/program.svg \var{\$(prog\_iconsdir)}
\tabul \var{\$(INSTALL\_DATA) \$(srcdir)}/metadata/icons/program-project.svg \var{\$(prog\_iconsdir)}
\tabul \var{\$(INSTALL\_DATA) \$(srcdir)}/metadata/icons/program-workspace.svg \var{\$(prog\_iconsdir)}
\tabul if [ ! -d \var{\$(prog\_mimedir)} ]; then \textbackslash
\tabul \tabul \var{\$(mkinstalldirs)} \var{\$(prog\_mimedir)}; \textbackslash
\tabul fi
\tabul \var{\$(INSTALL\_DATA) \$(srcdir)}/metadata/mime/program.xml \var{\$(prog\_mimedir)}
\tabul if [ ! -d \var{\$(prog\_metadir)} ]; then \textbackslash
\tabul \tabul mkdir -p \var{\$(prog\_metadir)}; \textbackslash
\tabul fi
\tabul \var{\$(INSTALL\_DATA) \$(srcdir)}/metadata/com.program.www.appdata.xml \var{\$(prog\_metadir)}
\tabul if [ ! -d \var{\$(prog\_desktopdir)} ]; then \textbackslash
\tabul \tabul mkdir -p \var{\$(prog\_desktopdir)}; \textbackslash
\tabul fi
\tabul appstream-util validate-relax --nonet \var{\$(prog\_metadir)}/com.program.www.appdata.xml
\tabul desktop-file-install --vendor="" \textbackslash
\tabul \tabul --dir=\var{\$(prog\_desktopdir)} -m 644 \textbackslash
\tabul \tabul \var{\$(prog\_desktopdir)}/program.desktop
\tabul touch -c \var{\$(prog\_iconsdir)}
\tabul if [ -u `which gtk-update-icon-cache` ]; then \textbackslash
\tabul \tabul gtk-update-icon-cache -q \var{\$(prog\_iconsdir)}; \textbackslash
\tabul fi
\tabul if [ -z "\var{\$(DESTDIR)}" ]; then \textbackslash
\tabul \tabul update-desktop-database \var{\$(prog\_desktopdir)} &> /dev/null || :; \textbackslash
\tabul \tabul update-mime-database \var{\$(prog\_datadir)}/mime &> /dev/null || :; \textbackslash
\tabul fi
\rtt{uninstall-local:}
\tabul -rm -rf \var{\$(prog\_pkgdatadir)}/data/*
\tabul -rmdir \var{\$(prog\_pkgdatadir)}/data
\tabul -rm -rf \var{\$(prog\_pkgdatadir)}/pixmaps/*
\tabul -rmdir \var{\$(prog\_pkgdatadir)}/pixmaps
\tabul -rm -rf \var{\$(prog\_pkgdatadir)}/pixmaps/*
\tabul -rmdir \var{\$(prog\_pkgdatadir)}
\tabul -rmdir \var{\$(prog\_docdir)}
\tabul -rm -rf \var{\$(prog\_iconsdir)}/program.svg
\tabul -rm -rf \var{\$(prog\_iconsdir)}/program-project.svg
\tabul -rm -rf \var{\$(prog\_iconsdir)}/program-workspace.svg
\tabul -rm -f \var{\$(prog\_desktopdir)}/program.desktop
\tabul -rm -f \var{\$(prog\_metadir)}/com.program.www.appdata.xml
\tabul touch -c \var{\$(prog\_iconsdir)}
\tabul if [ -u `which gtk-update-icon-cache` ]; then \textbackslash
\tabul \tabul gtk-update-icon-cache -q \var{\$(prog\_iconsdir)}; \textbackslash
\tabul fi
\tabul if [ -z "\var{\$(DESTDIR)}" ]; then \textbackslash
\tabul \tabul update-desktop-database \var{\$(prog\_desktopdir)} &> /dev/null || :; \textbackslash
\tabul \tabul update-mime-database \var{\$(prog\_datadir)}/mime &> /dev/null || :; \textbackslash
\tabul fi
\end{script}
}}

\subsection{The file: \bftt{src/Makefile.am}}
\label{makeall}

{\scriptsize{
\begin{script}
\bad{bin\_PROGRAMS} = prog

\dgtt{prog\_LDADD} = \var{\$(LGTK\_LIBS) \$(LIBXML2\_LIBS) \$(PANGOFT2\_LIBS) \$(FFMPEG\_LIBS) \$(GLU\_LIBS) \$(EPOXY\_LIBS)}

\dgtt{LIB\_CFLAGS} = \var{\$(LGTK\_CFLAGS) \$(LIBXML2\_CFLAGS) \$(PANGOFT2\_CFLAGS) \$(FFMPEG\_CFLAGS) \$(GLU\_CFLAGS) \$(EPOXY\_CFLAGS)}

if GTK3
  \dgtt{GTK\_VERSION}=\var{-DGTK3}
else
  \dgtt{GTK\_VERSION}=\var{-DGTK4}
endif

\dgtt{AM\_LDFLAGS} = 
\dgtt{AM\_CPPFLAGS} = \var{\$(GTK\_VERSION)}
\dgtt{AM\_FFLAGS} = 
\dgtt{AM\_CFLAGS} = \var{\$(LIB\_CFLAGS)}

\dgtt{prog\_fortran\_files} = file-1.f90 file-2.f90
\dgtt{prog\_fortran\_modules} = mod.f90

\dgtt{prog\_fortran} = \var{\$(prog\_fortran\_modules) \$(prog\_fortran\_files)}
\var{\$(patsubst \%.F90,\%.o,\$(prog\_fortran\_files)): \$(patsubst \%.F90,\%.o,\$(prog\_fortran\_modules))}

\dgtt{prog\_c} = main.c gui.c

\rtt{clean:}
\tabul -rm -f *.mod
\tabul -rm -f */*.o

\dgtt{prog\_SOURCES} = \var{\$(prog\_fortran)} \var{\$(prog\_c)}
\end{script}
}}


\newpage
\section{The CMake file(s)}

\subsection{The file: \bftt{CMakeList.txt}}
\label{cmakelist}
\vspace{-0.75cm}
{\notsotiny{
\begin{script}
\comm{CMake minimum configuration}
\confa{cmake\_minimum\_required} (\bftt{VERSION} 3.10)

\comm{Project version and details}
\confb{project}{\gtt{prog} \bftt{VERSION} 1.2.12}
\confb{project}{\gtt{prog} \bftt{DESCRIPTION} \magenta{"A tool box"}}
\confb{project}{\gtt{prog} \bftt{HOMEPAGE\_URL} \magenta{"https://www.program.com"}}
\confb{project}{\gtt{prog} \bftt{LANGUAGES} C Fortran}

\comm{Checking for pkgconf}
\confb{find\_package}{PkgConfig \bftt{REQUIRED}}
\comm{Checking for gtk+3.0:}
\confb{pkg\_check\_modules}{\magenta{GTK3} \bftt{REQUIRED} IMPORTED\_TARGET \dgtt{gtk3}}
\comm{Checking for libxml-2.0:}
\confb{pkg\_check\_modules}{\magenta{LIBXML2} \bftt{REQUIRED} IMPORTED\_TARGET \dgtt{libxml-2.0}}
\comm{Checking for libavcodec, and other FFMPEG based libraries:}
\confb{pkg\_check\_modules}{\magenta{LIBAVUTIL} \bftt{REQUIRED} IMPORTED\_TARGET \dgtt{libavutil}}
\confb{pkg\_check\_modules}{\magenta{LIBAVCODEC} \bftt{REQUIRED} IMPORTED\_TARGET \dgtt{libavcodec}}
\confb{pkg\_check\_modules}{\magenta{LIBAVFORMAT} \bftt{REQUIRED} IMPORTED\_TARGET \dgtt{libavformat}}
\confb{pkg\_check\_modules}{\magenta{LIBSWSCALE} \bftt{REQUIRED} IMPORTED\_TARGET \dgtt{libswscale}}
\comm{Checking for epoxy:}
\confb{pkg\_check\_modules}{\magenta{EPOXY} \bftt{REQUIRED} IMPORTED\_TARGET \dgtt{epoxy}}

\comm{Custom compiler flags}
\confb{set}{\confa{CMAKE\_C\_FLAGS} \magenta{"-O3"}}
\confb{set}{\confa{CMAKE\_Fortran\_FLAGS} \magenta{"-O3"}}
\comm{Compiler variables}
\confb{find\_package}{OpenMP}
\bbtt{if (}OpenMP\_FOUND\bbtt{)}
  \confb{set}{\confa{CMAKE\_C\_FLAGS} \varc{OpenMP\_C\_FLAGS}}
  \confb{set}{\confa{CMAKE\_Fortran\_FLAGS} \varc{OpenMP\_Fortran\_FLAGS}}
\bbtt{endif()}
\comm{Operating system:}
\bbtt{if (}LINUX\bbtt{)}
  \confb{add\_compile\_definitions}{LINUX}
\bbtt{elseif (}WINDOWS\bbtt{)}
  \confb{add\_compile\_definitions}{WINDOWS}
\bbtt{elseif (}APPLE\bbtt{)}
  \confb{add\_compile\_definitions}{OSX}
\bbtt{endif()}

\comm{Sources}
\confb{file}{GLOB \bftt{C\_SRC} RELATIVE \varc{CMAKE\_SOURCE\_DIR} \magenta{"src/*.c"}}
\confb{file}{GLOB \bftt{F\_SRC} RELATIVE \varc{CMAKE\_SOURCE\_DIR} \magenta{"src/*.f90"}}
\comm{Build rules}
\confb{add\_executable}{\gtt{prog} \varc{C\_SRC} \varc{F\_SRC}}
\comm{Header files include directories}
\confb{target\_include\_directories}{\gtt{prog} \bftt{PUBLIC} src}
\comm{Link with dependencies}
\confb{target\_link\_libraries}{\gtt{prog} \bftt{PUBLIC} PkgConfig::\magenta{GTK3}}
\confb{target\_link\_libraries}{\gtt{prog} \bftt{PUBLIC} PkgConfig::\magenta{LIBXML2}}
\confb{target\_link\_libraries}{\gtt{prog} \bftt{PUBLIC} PkgConfig::\magenta{LIBAVUTIL}}
\confb{target\_link\_libraries}{\gtt{prog} \bftt{PUBLIC} PkgConfig::\magenta{LIBAVCODEC}}
\confb{target\_link\_libraries}{\gtt{prog} \bftt{PUBLIC} PkgConfig::\magenta{LIBAVFORMAT}}
\confb{target\_link\_libraries}{\gtt{prog} \bftt{PUBLIC} PkgConfig::\magenta{LIBSWSCALE}}
\confb{target\_link\_libraries}{\gtt{prog} \bftt{PUBLIC} PkgConfig::\magenta{EPOXY}}

\comm{First import the GNU installation directory variables}
\confb{include}{\bftt{GNUInstallDirs}}
\comm{To install the main binary file:}
\confb{install}{\bftt{PROGRAMS} prog \bftt{DESTINATION} \varc{CMAKE\_INSTALL\_BINDIR}}
\comm{To install a directory, including all its content, recursively:}
\confb{install}{\bftt{DIRECTORY} data \bftt{DESTINATION} \varc{CMAKE\_INSTALL\_DATADIR}/prog \bftt{PATTERN} \magenta{"data/*"}}
\confb{install}{\bftt{DIRECTORY} pixmaps \bftt{DESTINATION} \varc{CMAKE\_INSTALL\_DATADIR}/prog \bftt{PATTERN}  \magenta{"pixmaps/*"}}
\confb{install}{\bftt{DIRECTORY} metadata/icons \bftt{DESTINATION} \varc{CMAKE\_INSTALL\_DATADIR}/pixmaps 
\bftt{                               PATTERN} \magenta{"metadata/icons/*.svg"}}
\comm{To install specific files:}
\confb{install}{\bftt{FILES} metadata/com.program.www.appdata.xml \bftt{DESTINATION} \varc{CMAKE\_INSTALL\_DATADIR}/metainfo}
\confb{install}{\bftt{FILES} metadata/program-mime.xml \bftt{DESTINATION} \varc{CMAKE\_INSTALL\_DATADIR}/mime/packages}
\confb{install}{\bftt{FILES} metadata/program.desktop \bftt{DESTINATION} \varc{CMAKE\_INSTALL\_DATADIR}/applications}

\comm{Running the post installation script:}
\confb{set}{\bftt{INSTALL\_DIR} "\varc{CMAKE\_INSTALL\_FULLDATADIR}"}
\confb{install}{\bftt{CODE} \magenta{"}\confb{execute\_process}{\bftt{COMMAND} \magenta{./post-install.sh} \varc{INSTALL\_DIR}}\magenta{"}}
\end{script}
}}

\subsection{The script: \bftt{post-install.sh}}
\label{poscript}

{\footnotesize{
\begin{script}
\bash

\comm{Retrieving the installation directory as first argument:}
\var{INSTALL\_DIR}=\dvar{1}

\comm{Setting up variables}
\var{prog\_metadir}=\dvar{\{INSTALL\_DIR\}}/metainfo
\var{prog\_desktopdir}=\dvar{\{INSTALL\_DIR\}}/applications
\var{prog\_iconsdir}=\dvar{\{INSTALL\_DIR\}}/pixmaps

\comm{Updating metadata information}
\bftt{appstream-util} validate-relax --nonet \dvar{\{prog\_metadir\}}/com.program.www.appdata.xml

\comm{Installing desktop file}
\bftt{desktop-file-install} --vendor=""  \textbackslash
                         --dir=\dvar{\{prog\_desktopdir\}} \textbackslash
                          -m 644 \dvar{\{prog\_desktopdir\}}/program.desktop

\comm{Updating icon file cache}
\bad{touch} --no-create \dvar{\{prog\_iconsdir\}}
\bif -u 'which gtk-update-icon-cache' \then
  \bftt{gtk-update-icon-cache} -q \dvar{\{prog\_iconsdir\}}
\bfi

\comm{Updating desktop database information}
\bftt{update-desktop-database} \dvar{\{prog\_desktopdir\}} &> /dev/null \pipe\pipe :;

\comm{Updating MIME database}
\bftt{update-mime-database} \dvar{\{prog\_datadir\}}/mime &> /dev/null \pipe\pipe :; 
\end{script}
}}
\newpage


\chapter{The packaging files}

\clearpage

\section{"\bftt{.spec}" file for RPM packaging}
\label{spec}
\vspace{-1cm}
{\scriptsize{
\begin{script}
\bad{Name:}      prog
\bftt{\%global} \var{upname} Program
\bad{Version:}   1.2.12
\bad{Release:}   3\var{\%\{?dist\}}
\bad{Summary:}   An nice program
\bad{License:}   AGPL-3.0-or-later
\bad{Source0:}   \gitauth/\%\{upname\}/archive/refs/tags/v\var{\%\{version\}}.tar.gz
\comm{Source1:   ./v\%\%\{version\}.tar.gz.asc}
\comm{Source2:   \%\%\{name\}.gpg}
\bad{URL:}        https://www.\var{\%\{name\}}.com/

\bad{BuildRequires:} make
\bad{BuildRequires:} automake
\bad{BuildRequires:} autoconf
\bad{BuildRequires:} pkgconf-pkg-config
\bad{BuildRequires:} gcc
\bad{BuildRequires:} gcc-gfortran
\bad{BuildRequires:} libgfortran
\bad{BuildRequires:} pkgconfig(gtk+-3.0)
\bad{BuildRequires:} pkgconfig(libxml-2.0)
\bad{BuildRequires:} pkgconfig(glu)
\bad{BuildRequires:} pkgconfig(epoxy)
\bad{BuildRequires:} pkgconfig(libavutil)
\bad{BuildRequires:} pkgconfig(libavcodec)
\bad{BuildRequires:} pkgconfig(libavformat)
\bad{BuildRequires:} pkgconfig(libswscale)
\bad{BuildRequires:} desktop-file-utils
\bad{BuildRequires:} libappstream-glib
\comm{BuildRequires: gnupg2}
\bad{Requires:} gtk3
\bad{Requires:} mesa-libGLU

\dgtt{\%prep}
\comm{\%\%\{gpgverify\} --keyring='\%\%\{SOURCE2\}' --signature='\%\%\{SOURCE1\}' --data='\%\%\{SOURCE0\}'}
\bftt{\%autosetup} -n \var{\%\{upname\}}-\var{\%\{version\}}

\dgtt{\%build}
\bftt{\%configure}
\bftt{\%make\_build}

\dgtt{\%install}
\bftt{\%make\_install}

\dgtt{\%check}
desktop-file-validate \var{\%\{buildroot\}}/\var{\%\{\_datadir\}}/applications/\var{\%\{name\}}.desktop
appstream-util validate-relax --nonet \var{\%\{buildroot\}}\var{\%\{\_metainfodir\}}/com.\var{\%\{name\}}.www.appdata.xml

\dgtt{\%files}
\bftt{\%license} COPYING
\var{\%\{\_bindir\}}/\var{\%\{name\}}
\var{\%\{\_datadir\}}/\var{\%\{name\}}
\var{\%\{\_datadir\}}/doc/\var{\%\{name\}}
\var{\%\{\_mandir\}}/man1/\var{\%\{name\}}.1*
\var{\%\{\_datadir\}}/applications/\var{\%\{name\}}.desktop
\var{\%\{\_metainfodir\}}/com.\var{\%\{name\}}.www.appdata.xml
\var{\%\{\_datadir\}}/mime/packages/\var{\%\{name\}}.xml
\var{\%\{\_datadir\}}/pixmaps/\var{\%\{name\}}.svg
\var{\%\{\_datadir\}}/pixmaps/\var{\%\{name\}}-program.svg
\var{\%\{\_datadir\}}/pixmaps/\var{\%\{name\}}-workspace.svg

\dgtt{\%changelog}
* Thu Mar 30 2023 Your Name <\email> - \gtt{1.2.12}-\rtt{3}
- Revised package: what you did here.
\end{script}
}}

\section{"\texttt{debian}" directory for Debian packaging}

%\subsection{The "\texttt{changelog}" file}

%\subsection{The "\texttt{control}" file}

\subsection{The "\texttt{copyright}" file}
\label{acopy}
{\notsotiny{
\input{copyright}
}}

\subsection{Example script to build and test locally your Debian package}
\label{btdebs}

{\tiny{
\begin{script}
\bash

\func{autoclean}
  \brm \violet{-f} aclocal.m4
  \brm \violet{-rf} auto4te.cache
  \brm \violet{-f} configure~
  aclocal
  autoconf
  automake \violet{--add-missing}
  \brm \violet{-f} configure~
\efunc

\var{VERSION}=\say{1.2.12}

\comm{Ensure that the top directory is clean}
\brm \violet{-f} \bap*\_\bap\dvar{VERSION}\bap*\bap
\bif \violet{-d} program-\dvar{VERSION} \then
  \brm \violet{-rf} program-\dvar{VERSION}
\bfi

\var{DOWN}\bad{=}\magenta{1}
\bif \dvar{DOWN} \beq \magenta{1} \then
  \comm{Retrieving the sources of the program from the official repository}
  \bad{wget} \gitprog/archive/refs/tags/v\dvar{VERSION}.tar.gz
  \bad{tar} \violet{-zxf} v\dvar{VERSION}.tar.gz
  \mv Program-\dvar{VERSION} program-\dvar{VERSION}
  \brm v\dvar{VERSION}.tar.gz
\bfi

\var{BUILD}\bad{=}\magenta{1}
\bif \dvar{BUILD} \beq \magenta{1} \then
  \cd program-\dvar{VERSION}
  \comm{If the 'debian' directory is not shipped with the official tarball}
  \comm{Uncomment and adapt the next line to add the 'debian' directory in the sources directory}
  \comm{cp -r ../debian-package-data debian}

  \comm{Ensure that there is a README file}
  \bcp README.md README

  \comm{To ensure that the version of the autotools files are suitable for this system}
  autoclean
  \comm{Debian packager identification}
  \bad{export} \dctt{DEBEMAIL}\bad{=}\say{\email}
  \bad{export} \dctt{DEBFULLNAME}\bad{=}\say{Your Name}
  \comm{Improved lintian command for package verification}
  \bad{alias} \dctt{lintian}\bad{=}\reg{lintian -EviIL +pedantic --color auto}
  \comm{Creating the Debian origin tarball}
  \bftt{dh\_make} \violet{--createorig -s -y}
  \comm{Building the Debian package}
  \bftt{dpkg-buildpackage}
  \cd ..
\bfi

\var{TEST}\bad{=}\magenta{1}
\bif \dvar{TEST} \beq \magenta{1} \then
  \comm{Linitian on 'program.changes'}
  \lint ./program\_*changes \bad{>&} results.lintian
  \cd program-\dvar{VERSION}
  \comm{Licensecheck to check for license}
  \bftt{licensecheck} \violet{--recursive --copyright} \btt{.} \bad{>&} ../license.check
  \comm{Scan-copyrights to create a copyrights file from scratch using licensecheck}
  \bftt{scan-copyrights} \bad{>&} ../scan.copy
  \cd ..
\bfi
 
\var{PIUPARTS}\bad{=}\magenta{0}
\bif \dvar{PIUPARTS} \beq \magenta{1} \then
  \echo \say{Piuparts on program.deb} \bad{>} results.piuparts
  \echo \say{ } \bad{>>} results.piuparts
  \piup ./program\_\dvar{VERSION}*.deb \bad{&>>} results.piuparts
  \echo \say{ } \bad{>>} results.piuparts
  \echo \say{Piuparts on program-data.deb} \bad{>>} results.piuparts
  \echo \say{ } \bad{>>} results.piuparts
  \piup ./program-data*.deb \bad{&>>} results.piuparts
\bfi
\end{script}

}}

\subsection{Example of ITP bug report message}
\label{bugreport}

{\footnotesize{
\begin{script}
\magenta{Subject: ITP: }\bftt{program} \magenta{--} \bftt{short description}
\bad{Package: wnpp}
\bad{X-Debbugs-Cc:} \magenta{debian-devel@lists.debian.org}
\bad{Owner: }\bftt{Your Name \magenta{<\email>}}
\bad{Severity: wishlist}

* Package name    : \bftt{program}
  Version         : \bftt{1.2.12}
  Upstream Author : \bftt{Your Name \magenta{<\email>}}
* URL             : \red{https://www.program.com/}
* License         : AfferoGPLv3+
  Programming Lang: C, Fortran90, GLSL
  Description     : \bftt{short description}

"program" is a cross-platform software doing many things, 
and I will elaborate about the most significant achievements now ...

Among the many reasons I could think of, to help you understand why you
should consider "program" seriously, I will pick 3: 

1) This is the first reason ...

2) This is the second reason ...

3) This is the third reason ...

For the time being "program" is being maintained by ..

Sources are hosted on Github and Salsa:

"program" sources:          \gitprog
"program" sources on Salsa: https://salsa.debian.org/author/program

Finally I am looking for a sponsor. 
Indeed as this is my first Debian package, I am looking for all the help I can get. 
I am willing to spend time to learn how to do things properly 
to get "program" approved by the Debian community and ultimately packaged. 
If the proper way to do that is inside a packaging team then after checking 
the packaging teams at \href{https://wiki.debian.org/Teams}{https://wiki.debian.org/Teams} 
I think the "Debian *** Team" is likely to be the most appropriate place to start.

Best regards.

Your Name
\end{script}
}}

\section{Metadata for Linux intergation}

\subsection{Custom MIME file(s) setup}
\label{mime}

This file is required to define file association(s) and uses the \texttt{XML} file format:
{\scriptsize{
\begin{script}
\dbtt{<?xml}? \abtt{version=}\red{"1.0"} \abtt{encoding=}\red{"UTF-8"}\dbtt{?>}
\dbtt{<mime-info} \abtt{xmlns=}\red{"http://www.freedesktop.org/standards/shared-mime-info"}\dbtt{>}
  \dbtt{<mime-type} \abtt{type=}\red{"application/x-ppf"}\dbtt{>}
    \dbtt{<comment>}Program Project file\dbtt{</comment>}
    \dbtt{<glob} \abtt{pattern=}\red{"*.ppf"}\dbtt{/>}
    \dbtt{<generic-icon} \abtt{name=}\red{"program-project"}\dbtt{/>}
    \dbtt{<sub-class-of} \abtt{type=}\red{"text/plain"}\dbtt{/>}
  \dbtt{</mime-type>}
  \dbtt{<mime-type} \abtt{type=}\red{"application/x-pwf"}\dbtt{>}
    \dbtt{<comment>}Program Workspace file\dbtt{</comment>}
    \dbtt{<glob} \abtt{pattern=}\red{"*.pwf"}\dbtt{/>}
    \dbtt{<generic-icon} \abtt{name=}\red{"program-workspace"}\dbtt{/>}
    \dbtt{<sub-class-of} \abtt{type=}\red{"application/octet-stream"}\dbtt{/>}
  \dbtt{</mime-type>}
\dbtt{</mime-info>}
\end{script}
}}
\\
\noindent Each file association is defined between the \dbtt{<mime-type>}  ... \dbtt{</mime-type>} tags:
{\scriptsize{
\begin{script}
...
  \dbtt{<mime-type} \abtt{type=}\red{"application/x-ppf"}\dbtt{>}
    \dbtt{<comment>}Program Project file\dbtt{</comment>}
    \dbtt{<glob} \abtt{pattern=}\red{"*.ppf"}\dbtt{/>}
    \dbtt{<generic-icon} \abtt{name=}\red{"program-project"}\dbtt{/>}
    \dbtt{<sub-class-of} \abtt{type=}\red{"text/plain"}\dbtt{/>}
  \dbtt{</mime-type>}
... 
\end{script}
}}
\noindent Where the tags are used as follow:
\begin{itemize}
\item \dbtt{<comment>}: to describe the file format.
\item \dbtt{<glob}: to define the file extension.
\item \dbtt{<generic-icon}: to define the icon name, without extension (svg, png). \\
In the example providing that the file "\texttt{program-project.svg}" exists and is located in "\texttt{/usr/share/pixmaps}", 
then it is enough for the system to find it and associate it with the file format.
\item \dbtt{<sub-class-of}: to help the system recognize the format:
\begin{itemize}
\item \red{\texttt{"text/plain"}} for text formats.
\item \red{\texttt{"application/octet-stream"}} for binary formats.
\end{itemize}
\end{itemize}
\noindent To get information about the format of \texttt{this\_file}: 
\vspace{-0.25cm}
\begin{script}
\fprompt{~} \bftt{gio} \rtt{info} this\_file
\end{script}

\subsection{Desktop entry for desktop application}
\label{desktop}

The following is an example of the content of a installed "\bftt{*.desktop}" file. 
{\scriptsize{
\begin{script}
\dgtt{[Desktop Entry]}
\dctt{Version}=\violet{1.0}
\dctt{Name}=Program
\dctt{GenericName}=Program
\dctt{Comment}=For more information: https://www.program.com/
\dctt{Comment}\aqua{[fr]}=Pour plus d'information: https://www.program.com/
\dctt{Comment}\aqua{[es]}=Para más información: https://www.program.com/
\dctt{Keywords}=chemistry;physics;
\dctt{Keywords}\aqua{[fr]}=chimie;physique;
\dctt{Keywords}\aqua{[es]}=química;física;
\dctt{Exec}=program \violet{\%F}
\comm{Providing that the file "program.svg" exists and is located in "/usr/share/pixmaps"}
\comm{Then it is enough for the system to find it and associate it using:}
\dctt{Icon}=propram
\dctt{Terminal}=\violet{false}
\dctt{Type}=Application
\dctt{MimeType}=application/x-ppf;application/x-pwf;
\dctt{StartupNotify}=\violet{true}
\dctt{Categories}=Education;Science;Chemistry;Physics;
\end{script}
}}
\\
\noindent For \dctt{MimeType} the keywords must match the values defined in the MIME association file by the \dbtt{<mime-type }\abtt{type=}\red{"keyword"}\dbtt{>}
 opening tags, see [Sec.~\ref{mime}].

\clearpage
\subsection{AppStream metadata for desktop application}
\label{appstream}
\vspace{-0.5cm}
{\notsotiny{
\begin{script}
\dbtt{<?xml}? \abtt{version=}\red{"1.0"} \abtt{encoding=}\red{"UTF-8"}\dbtt{?>}
<!-- This is a commented line for a XML file -->
<!-- Copyright 2023 Your Name -->
<!-- The type of desktop object described -->
\dbtt{<component} \abtt{type=}\red{"desktop-application"}\dbtt{>}

  \dbtt{<id>}com.program.www\dbtt{</id>}
  \dbtt{<name>}Program\dbtt{</name>}
  \dbtt{<summary>}An interesting program\dbtt{</summary>}
  <!-- The license of your program --> 
  \dbtt{<project\_license>}AGPL-3.0-or-later\dbtt{</project\_license>}
  \dbtt{<metadata\_license>}FSFAP\dbtt{</metadata\_license>}
  
  \dbtt{<description>}
    \dbtt{<p>}Program: an utility to do many things !\dbtt{</p>}
  \dbtt{</description>}
  <!-- The desktop file that describes launch information --> 
  \dbtt{<launchable} \abtt{type=}\red{"desktop-id"}\dbtt{>}program.desktop\dbtt{</launchable>}

  <!-- Up to 3 screenshot(s), used in the Linux app store to illustrate your program -->
  \dbtt{<screenshots}
    \dbtt{<screenshot>}
      <!-- The video cannot be the default screenshot -->
      <!-- Container: mkv, Video codecs allowed: av1 or vp9, Audio codec allowed: opus -->
      \dbtt{<caption>}A nice video to illustrate the features of the program\dbtt{</caption>}
      \dbtt{<video} \abtt{container=}\red{"webm"} \abtt{codec=}\red{"vp9"} \abtt{width=}\red{"1920"} \abtt{height=}\red{"1080"}\dbtt{>}https://www.program.com/my-video.webm\dbtt{</video>}
    \dbtt{</screenshot}
    \dbtt{<screenshot} \abtt{type=}\red{"default"}\dbtt{>}
      \dbtt{<caption>}Overview of the program\dbtt{</caption>}
      \dbtt{<image} \abtt{type=}\red{"source"} \abtt{width=}\red{"1600"} \abtt{height=}\red{"900"}\dbtt{>}https://www.program.com/program-overview.png\dbtt{</image>}
    \dbtt{</screenshot}
    \dbtt{<screenshot>}
      \dbtt{<caption>}A detail of the program\dbtt{</caption>}
      \dbtt{<image} \abtt{type=}\red{"source"} \abtt{width=}\red{"1600"} \abtt{height=}\red{"900"}\dbtt{>}https://www.program.com/program-detail.png\dbtt{</image>}
    \dbtt{</screenshot}
    \dbtt{<screenshot>}
      \dbtt{<caption>}Another detail of the program\dbtt{</caption>}
      \dbtt{<image} \abtt{type=}\red{"source"} \abtt{width=}\red{"1600"} \abtt{height=}\red{"900"}\dbtt{>}https://www.program.com/program-other-detail.png\dbtt{</image>}
    \dbtt{</screenshot}
  \dbtt{</screenshots}
  
  \dbtt{<url} \abtt{type=}\red{"homepage"}\dbtt{>}https://www.program.com/\dbtt{</url>}
  \dbtt{<url} \abtt{type=}\red{"bugtracker"}\dbtt{>}\gitprog/issues/new/choose\dbtt{</url>}
  \dbtt{<developer\_name>}Mr. Your Name\dbtt{</developer\_name>}
  \dbtt{<update\_contact>}\metamail\dbtt{</update\_contact>}
  
  \dbtt{<provides>}
    \dbtt{<binary>}program\dbtt{</binary>}
  \dbtt{</provides>}
 
  \dbtt{<keywords>}
    \dbtt{<keyword>}chemistry\dbtt{</keyword>}
    \dbtt{<keyword>}physics\dbtt{</keyword>}
    \dbtt{<keyword} \abtt{xml:lang=}\red{"fr"}\dbtt{>}chimie\dbtt{</keyword>}
    \dbtt{<keyword} \abtt{xml:lang=}\red{"fr"}\dbtt{>}physique\dbtt{</keyword>}
    \dbtt{<keyword} \abtt{xml:lang=}\red{"es"}\dbtt{>}química\dbtt{</keyword>}
    \dbtt{<keyword} \abtt{xml:lang=}\red{"es"}\dbtt{>}física\dbtt{</keyword>}
  \dbtt{</keywords>}

  \dbtt{<content\_rating type}=\red{"oars-1.1}" \dbtt{/>}
  
  \dbtt{<releases>}
    \dbtt{<release} \abtt{version=}\red{"1.2.12"} \abtt{date=}\red{"2023-03-29"}\dbtt{>}
      \dbtt{<description>}
        \dbtt{<p>}Release of version 1.2.12:\dbtt{</p>}
        \dbtt{<p>}- Bug corrections, for details see: \dbtt{<url>}\gitprog/releases/tag/v1.2.12\dbtt{</url>}\dbtt{</p>}
        \dbtt{<p>}- Improvements, for details see: \dbtt{<url>}\gitprog/releases/tag/v1.2.12\dbtt{</url>}\dbtt{</p>}
      \dbtt{</description>}
    \dbtt{</release>}	  
    \dbtt{<release} \abtt{version=}\red{"1.2.11"} \abtt{date=}\red{"2023-02-06"}\dbtt{>}
      \dbtt{<description>}
        \dbtt{<p>}Release of version 1.2.11:\dbtt{</p>}
        \dbtt{<p>}- Bug corrections, for details see: \dbtt{<url>}\gitprog/releases/tag/v1.2.11\dbtt{</url>}\dbtt{</p>}
      \dbtt{</description>}
    \dbtt{</release>}
  \dbtt{</releases>}
  
\dbtt{</component>}
\end{script}
}}
