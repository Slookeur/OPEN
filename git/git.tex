\chapter{Version management using Git}

\section{Introduction}

\href{https://git-scm.com}{Git} is a distributed version control system that allows you to keep track of changes to your code over time. 
It is widely used among programmers collaboratively developing source code. \\[0.25cm]
Git was originally developed by Linus Torvald in 2005 to improve the collaboration of coders working on the Linux kernel. \\[0.25cm]
Basically Git offers a simple command line interface, several GUI frontends being available, 
that allows you keep track easily of the modification(s) on the project file(s) that you are monitoring. \\
The next sections will introduce a limited number of Git features to present the basics required to manage a project using Git. \\
For more complete documentation:
\begin{itemize}
\item The official Git documentation: \href{https://git-scm.com/doc}{https://git-scm.com/doc}
\item The reference manual: \href{https://git-scm.com/docs}{https://git-scm.com/docs}
\item The Pro Git Book: \href{https://git-scm.com/book/en/v2}{https://git-scm.com/book/en/v2}
\end{itemize}

\section{Installing Git}

Before to start using Git, it is required to install it on your computer (if not done already). 
For Linux use your favorite software management utility: 
\begin{itemize}
\item Red Hat based Linux (using the \bftt{dnf} command):
\begin{scripti}
\fprompt{~} sudo \bftt{dnf} install git
\end{scripti}
\item Debian based Linux (using the \bftt{apt} command):
\begin{scripti}
\uprompt{~} sudo \bftt{apt} install git
\end{scripti}
\end{itemize}
To learn more about software installation on Linux check out my \href{https://www.ipcms.fr/wp-content/uploads/2021/11/linux.pdf}{Linux tutorial}. \\
To learn more about scripting and the usage of the command line check out my \href{https://www.ipcms.fr/wp-content/uploads/2021/05/bash.pdf}{Bash tutorial}.\\[0.25cm]
For Windows and Mac download the latest version of Git from the official Git website (\href{https://git-scm.com/downloads}{https://git-scm.com/downloads}) 
and follow the installation instructions provided on the website for your operating system.

\section{Configuring Git}
\label{gitconfig}
Once Git is installed, you need to configure it with your name and email address:
Open a terminal and enter the following commands:
\begin{script}
\fprompt{~} \bftt{git} \rtt{config} \blue{--global} user.name "\abtt{Your Name}"
\fprompt{~} \bftt{git} \rtt{config} \blue{--global} user.email \dctt{\email}
\end{script}
\\[-0.25cm]
\noindent Replace \abtt{Your Name} and \dctt{\email} with your own. \\
The effect of the "\blue{\texttt{-{-}global}}" option is that \abtt{Your Name} and \dctt{\email} will be used as credentials by default for all your Git repositories. \\
However you might need to work on repositories where name and email could be different, in that case use the same command line without the "\blue{\texttt{-{-}global}}" option, 
but only after the initialization of the working repository (see bellow).

\section{Creating a Git repository}

To start using Git, you need to create a Git repository. 
A Git repository is a folder that contains your source code, Git will keep track of changes to your code within this folder. 
To create a Git repository, navigate to the folder where you want to store your code, and use:
\begin{script}
\fprompt{~} cd program
\fprompt{~/program} \bftt{git} \rtt{init}
\end{script}
\\[-0.25cm]
\noindent This will create a new Git repository in the current directory. \\
You can check that the repository has been created by listing the hidden files in the active folder: 
\begin{script}
\fprompt{~/program} ls -hla 
drwxr-xr-x.  8 leroux dmo 4,0K 27 mars  21:58 \btt{.git}
\end{script}
\\
\noindent This \bftt{.git} folder will contain all the versioning information of your project. \\
If you erase this folder you erase all project history, also called index, 
and the only information that will remain consist of the current file(s) and folder(s) in the active directory. 

\section{Adding a branch to the project}

Git projects can be seen as tree view, made with branches. The "\bftt{git} \rtt{init}" command creates a first branch called "\texttt{master}" 
to allow you to start working immediately. 
However you will likely need to create your own branches, in particular for web hosting proposes (see chapter~\ref{hosting})
where main branches are usually called "\texttt{Main}" or "\texttt{main}". \\[0.25cm]
To create a branch use: 
\begin{script}
\fprompt{~/program} \bftt{git} \rtt{branch} \btt{-m} Main
\end{script} \\[-0.75cm]
\noindent To switch to another branch: 
\begin{script}
\fprompt{~/program} \bftt{git} \rtt{checkout} branch
\end{script} \\[-0.75cm]
\noindent Where the last option on the line, in this example "\texttt{branch}", is the name of the target branch to switch to. \\
You can also create a branch and immediately switch to that branch:
\begin{script}
\fprompt{~/program} \bftt{git} \rtt{checkout} -b branch
\end{script} \\[-0.75cm]
\noindent The latest actually check if "\texttt{branch}" exists, if it does not create it, then switch to the target branch. \\
In the following the sentence "adding file(s) to your Git repository" will be used extensively, it will be more accurate to write 
"adding file(s) to target branch of your Git repository". 

\section{Adding files to the project index}

Now that you have a Git repository, and that you created your main branch, you can add file(s) to your Git repository, using:
\begin{script}
\fprompt{~/program} \bftt{git} \rtt{add} this.file
\end{script}
\\[-0.25cm]
For example to add the file \btt{README.md}, use: 
\begin{script}
\fprompt{~/program} \bftt{git} \rtt{add} \btt{README.md}
\end{script}
\\[-0.25cm]
\noindent To add all file(s) and folder(s) in the active folder, use:
\begin{script}
\fprompt{~/program} \bftt{git} \rtt{add} \btt{.}
\end{script}
\\[-0.5cm]
\noindent The \bftt{git} \rtt{add} command is used to add file contents to the project index, it updates the index using files 
found in the working directory tree: 
\begin{itemize}
\item It is required to use the \rtt{add} command to add any newly created, modified or deleted files to the index. 
\item This command can be performed several times before a commit. 
\item The \bftt{git} \rtt{status} command can used to obtain a summary of the files that have been changed and that staged for the next commit:
{\scriptsize{
\begin{scripti}
\fprompt{~} mkdir program
\fprompt{~} cd program
\fprompt{~/program} \bftt{git} init
\brown{astuce: Utilisation de 'master' comme nom de la branche initiale. Le nom de la branche
astuce: par défaut peut changer. Pour configurer le nom de la branche initiale
astuce: pour tous les nouveaux dépôts, et supprimer cet avertissement, lancez :
astuce: 
astuce: 	git config --global init.defaultBranch <nom>
astuce: 
astuce: Les noms les plus utilisés à la place de 'master' sont 'main', 'trunk' et
astuce: 'development'. La branche nouvellement créée peut être rénommée avec :
astuce: 
astuce: 	git branch -m <nom>}
Dépôt Git vide initialisé dans ~/program/.git/
\fprompt{~/program} cp ../README.md .
\fprompt{~/program} \bftt{git} \rtt{add} \btt{README.md}
\fprompt{~/program} \bftt{git} \rtt{status}
Sur la branche master

Aucun commit

Modifications qui seront validées :
  (utilisez "git rm --cached <fichier>..." pour désindexer)
	\dgtt{nouveau fichier : README.md}
\end{scripti}
}}
\end{itemize}
\section{Committing changes}

Now that you have added file(s) to your Git repository, you can {\bf{commit}} the changes. 
That means apply the changes to the repository and create a commit: a snapshot of the current state of your project/code history recorded by Git. \\
To commit changes, use the following command:
\begin{script}
\fprompt{~/program} \bftt{git} \rtt{commit} \btt{-m} "Commit message"
\end{script}
\\[-0.25cm]
\noindent Replace \texttt{"Commit message"} with a brief description of the changes you made, ex:
\begin{script}
\fprompt{~/program} now=`date`
\fprompt{~/program} version="1.0"
\fprompt{~/program} \bftt{git} \rtt{commit} \btt{-m}  "Program [V-\var{\$version}] \var{\$now}"
\end{script}

\section{Checking commit history}

To view the commit history of a Git repository, use the following command:
\begin{script}
\fprompt{~/program} \bftt{git} \rtt{log}
\end{script}
\\[-0.25cm]
\noindent This will display a list of all the commits that have been made to the repository, 
along with their commit messages and other information.

\section{Using Git to work on a remote, on-line, repository}

The main interest of Git is to help you manage collaborative work(s) on remote, on-line, repositories, hosted on platforms like \href{https://github.com}{GitHub} or \href{https://gitlab.com}{GitLab}. 
Furthermore it is almost mandatory to host your project on such platforms if you intent to distribute it using the pipelines of the open source community. \\
The next chapter will present how to host a project on these websites, then it introduce the basic Git commands to use to manage such a project, see [Sec.~\ref{onlinegit}] for details. 

\section{Conclusion}

At this point you should have gained basic knowledge of Git a powerful tool for managing source codes essential to any developer. 
This was the second prerequisite to be able take advantage of the package delivery system of the open source community to distribute your software.

