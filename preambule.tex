
\usepackage[utf8]{inputenc}
\usepackage[T1]{fontenc}
\usepackage{babel}

\usepackage{textcomp}
\usepackage{pifont}
% Les polices possibles (pour tout le document)
%\usepackage[bitstream-charter]{mathdesign}

%Font ps pour le pdf
\usepackage{pslatex}
%Hyperliens pour le pdf
\usepackage[pdfauthor   = {Sébastien\ Le\ Roux},
    pdftitle = {From\ the\ source\ code\ to\ free\ software\ distribution:\ review\ and\ tips},
    pdfsubject = {From\ the\ source\ code\ to\ software\ distribution},
    pdfkeywords = {Linux\ GNU\ Make\ RPM\ DEB\ Git\ Github},
    pdfcreator = {Latex+DviPDF},
    pdfproducer = {LaTeX+DviPDF},
    pdfstartview=FitV   % Ouverture avec ajustement de l'image
    dvips=true,         % Use hyperref with dvips
    colorlinks=true,    % Lien hypertext en couleur
    plainpages=false,   %
    pagebackref=true,   % Permet d'ajouter des liens retour dans la biblio ...
    backref=page,       % .. ces liens pointent vers les 'pages' des citations
    hyperindex=true,    % Ajoute des liens dans l'index
    linktocpage=true,   % Lien sur les numéros de page et non le text 
    breaklinks=true,    % Permet le retour à la ligne dans les liens trop longs
    urlcolor= blue,     % Couleur des liens externes
%    linkcolor= black,   % Couleur des liens internes
    bookmarks=true,     % Création des signets pour Acrobat
    bookmarksopen=false % Toute l'arborescence est dépliée à l'ouverture
]{hyperref}
%\usepackage[pageref]{backref}

%Insertion d'images
\usepackage{graphicx}
\DeclareGraphicsExtensions{.eps}

%Dessins latex
% Brouillon ? = afficher les images "false" ou seulement les cadres "true"
% effet sur tout le document
\newcommand{\ddst}{false}
%\usepackage{picins}

\usepackage[x11names]{xcolor}

% Mode verbatim avancé
\usepackage{alltt}

% Système de liste/énumération
\usepackage{pifont}
\usepackage{enumerate}
%\usepackage{enumitem}

% Ecriture des mathématiques
\usepackage{amsmath}
\usepackage{amssymb}
\usepackage{amscd}
\usepackage{theorem}

% tableaux
\usepackage{hhline}
%\usepackage{array}
\usepackage{multirow}
\usepackage{tabls}
%\usepackage{bbding}

%% Mise en page
%\voffset         0.0cm
%\hoffset         0.0cm
%\textheight     23.0cm
%\textwidth      16.0cm
%\topmargin      -0.5cm
%\oddsidemargin   0.0cm
%\evensidemargin  0.0cm

% pages en landscape dans document portrait
\usepackage{lscape}

% Aspect de pages
\usepackage{setspace}
%\onehalfspacing
%\doublespacing
%\setstretch{3}

%\usepackage{fancyhdr,fancybox}
%\fancyhead{}
%\fancyhead[RO]{\scriptsize{\slshape\rightmark}}
%\fancyhead[LO]{\scriptsize{\thepage}}
%\fancyhead[RE]{\scriptsize{\thepage}}
%\fancyhead[LE]{\scriptsize{\slshape\leftmark}}
%\pagestyle{fancy}

% Numérotation pour les sous-sous-sections
\setcounter{secnumdepth}{3}
% Table des matières/figures/tables par chapitre
%\usepackage{minitoc}
% Pour placer des notes de bas de pages dans les titres
%\usepackage[stable]{footmisc}

%Définitions de théorèmes
\theoremstyle{plain}
\theoremheaderfont{\scshape}
\theorembodyfont{\normalfont\itshape}
\newtheorem{HK}{Théorème}

% notsotiny font size
\makeatletter
\newcommand\notsotiny{\@setfontsize\notsotiny{6.5}{7.6878}}
\newcommand\notsofoot{\@setfontsize\notsofoot{9.5}{11.6}}
\makeatother

% Créer un environnement résumé
\def\abstract{
   \begin{center}
   \begin{minipage}{12cm}
   \begin{center}{\bf Résumé}\end{center}\par\small}
\def\endabstract{\par\end{minipage}\end{center}\vspace{1cm}}

% Bilblio:
% Bilblio:
\newcommand{\aap}{Astron. \& Astrophys.}
\newcommand{\aasup}{Astron. \& Astrophys. Suppl. Ser.}
\newcommand{\aj}{Astron. J.}
\newcommand{\aph}{Acta Phys.}
\newcommand{\act}{Acta Cryst.}
\newcommand{\actc}{Acta Cryst.}
\newcommand{\acta}{Acta Cryst. A}
\newcommand{\actb}{Acta Cryst. B}
\newcommand{\advp}{Adv. Phys.}
\newcommand{\ajp}{Amer. J. Phys.}
\newcommand{\ajm}{Amer. J. Math.}
\newcommand{\amsci}{Amer. Sci.}
\newcommand{\anofd}{Ann. Fluid Dyn.}
\newcommand{\am}{Ann. Math.}
\newcommand{\ap}{Ann. Phys. (NY)}
\newcommand{\adp}{Ann. Phys. (Leipzig)}
\newcommand{\ao}{Appl. Opt.}
\newcommand{\apl}{Appl. Phys. Lett.}
\newcommand{\app}{Astroparticle Phys.}
\newcommand{\apj}{Astrophys. J.}
\newcommand{\apjsup}{Astrophys. J. Suppl.}
\newcommand{\apss}{Astrophys. Space Sci.}
\newcommand{\araa}{Ann. Rev. Astron. Astrophys.}
\newcommand{\baas}{Bull. Amer. Astron. Soc.}
\newcommand{\baps}{Bull. Amer. Phys. Soc.}
\newcommand{\cmp}{Comm. Math. Phys.}
\newcommand{\cpam}{Commun. Pure Appl. Math.}
\newcommand{\cppcf}{Comm. Plasma Phys. \& Controlled Fusion}
\newcommand{\cpc}{Comp. Phys. Comm.}
\newcommand{\cms}{Comp. Mat. Sci.}
\newcommand{\cqg}{Class. Quant. Grav.}
\newcommand{\cra}{C. R. Acad. Sci. A}
\newcommand{\crv}{Chem. Rev.}
\newcommand{\cp}{Chem. Phys.}
\newcommand{\cma}{Chem. Mater.}
\newcommand{\epl}{Eur. Phys. Lett.}
\newcommand{\ejm}{Eur. J. Mineral.}
\newcommand{\fed}{Fusion Eng. \& Design}
\newcommand{\ft}{Fusion Tech.}
\newcommand{\grg}{Gen. Relativ. Gravit.}
\newcommand{\ieeens}{IEEE Trans. Nucl. Sci.}
\newcommand{\ieeeps}{IEEE Trans. Plasma Sci.}
\newcommand{\ijimw}{Interntl. J. Infrared \& Millimeter Waves}
\newcommand{\ip}{Infrared Phys.}
\newcommand{\irp}{Infrared Phys.}
\newcommand{\jap}{J. Appl. Phys.}
\newcommand{\jac}{J. Appl. Cryst.}
\newcommand{\jacs}{J. Am. Chem. Soc.}
\newcommand{\jamcs}{J. Am. Ceram. Soc.}
\newcommand{\jasa}{J. Acoust. Soc. America}
\newcommand{\jcp}{J. Chem. Phys.}
\newcommand{\jcop}{J. Comp. Phys.}
\newcommand{\jetp}{Sov. Phys.--JETP}
\newcommand{\jetpl}{JETP Lett.}
\newcommand{\jfe}{J. Fusion Energy}
\newcommand{\jfm}{J. Fluid Mech.}
\newcommand{\jmp}{J. Math. Phys.}
\newcommand{\jne}{J. Nucl. Energy}
\newcommand{\jnec}{J. Nucl. Energy, C: Plasma Phys., Accelerators, Thermonucl. Res.}
\newcommand{\jncs}{J. Non-Cryst. Solids.}
\newcommand{\jnm}{J. Nucl. Mat.}
\newcommand{\joam}{J. Optoelect. Adv. Mat.}
\newcommand{\jpcssp}{J. Phys. C: Solid State Phys.}
\newcommand{\jpc}{J. Phys. Chem.}
\newcommand{\jpcs}{J. Phys. Chem. Sol.}
\newcommand{\jpcm}{J. Phys.: Cond. Mat.}
\newcommand{\jpp}{J. Plasma Phys.}
\newcommand{\jpsj}{J. Phys. Soc. Japan}
\newcommand{\jqc}{J. Quant. Chem.}
\newcommand{\jssc}{J. Sol. Stat. Chem.}
\newcommand{\jsi}{J. Sci. Instrum.}
\newcommand{\jvst}{J. Vac. Sci. \& Tech.}
\newcommand{\mcp}{Mat. Chem. Phys.}
\newcommand{\molp}{Mol. Phys.}
\newcommand{\nat}{Nature}
\newcommand{\nature}{Nature}
\newcommand{\nedf}{Nucl. Eng. \& Design/Fusion}
\newcommand{\nf}{Nucl. Fusion}
\newcommand{\nim}{Nucl. Inst. \& Meth.}
\newcommand{\nimpr}{Nucl. Inst. \& Meth. in Phys. Res.}
\newcommand{\np}{Nucl. Phys.}
\newcommand{\npb}{Nucl. Phys. B}
\newcommand{\ntf}{Nucl. Tech./Fusion}
\newcommand{\npbpc}{Nucl. Phys. B (Proc. Suppl.)}
\newcommand{\inc}{Nuovo Cimento}
\newcommand{\nc}{Nuovo Cimento}
\newcommand{\pcg}{Phys. Chem. Glasses}
\newcommand{\pf}{Phys. Fluids}
\newcommand{\pfa}{Phys. Fluids A: Fluid Dyn.}
\newcommand{\pfb}{Phys. Fluids B: Plasma Phys.}
\newcommand{\pl}{Phys. Lett.}
\newcommand{\pla}{Phys. Lett. A}
\newcommand{\plb}{Phys. Lett. B}
\newcommand{\prep}{Phys. Rep.}
\newcommand{\pnas}{Proc. Nat. Acad. Sci. USA}
\newcommand{\pp}{Phys. Plasmas}
\newcommand{\ppcf}{Plasma Phys. \& Controlled Fusion}
\newcommand{\phitrsl}{Philos. Trans. Roy. Soc. London}
\newcommand{\plmb}{Phil. Mag. B} 
\newcommand{\pml}{Phil. Mag. Lett.}
\newcommand{\pmm}{Phil. Mag.}
\newcommand{\prl}{Phys. Rev. Lett.}
\newcommand{\pr}{Phys. Rev.}
\newcommand{\physrev}{Phys. Rev.}
\newcommand{\pra}{Phys. Rev. A}
\newcommand{\prb}{Phys. Rev. B}
\newcommand{\prc}{Phys. Rev. C}
\newcommand{\prd}{Phys. Rev. D}
\newcommand{\pre}{Phys. Rev. E}
\newcommand{\ps}{Phys. Scripta}
\newcommand{\pstb}{Phys. Stat. Sol. b}
\newcommand{\procrsl}{Proc. Roy. Soc. London}
\newcommand{\rmp}{Rev. Mod. Phys.}
\newcommand{\rsi}{Rev. Sci. Inst.}
\newcommand{\rpp}{Rep. Prog. Phys.}
\newcommand{\science}{Science}
\newcommand{\sciam}{Sci. Am.}
\newcommand{\susc}{Surf. Sci}
\newcommand{\sam}{Stud. Appl. Math.}
\newcommand{\sjpp}{Sov. J. Plasma Phys.}
\newcommand{\spd}{Sov. Phys.--Doklady}
\newcommand{\sptp}{Sov. Phys.--Tech. Phys.}
\newcommand{\spu}{Sov. Phys.--Uspeki}
\newcommand{\skt}{Sky and Telesc.}
\newcommand{\ssi}{Solid State Ionics}
\newcommand{\ssc}{Solid State Com.}
\newcommand{\ssnmr}{Solid State Nuc. Mag. Res.}
\newcommand{\zfp}{Zs. f. Phys.}
\newcommand{\zk}{Z. Kristallogr.}


%\usepackage{chapterbib}

\usepackage[square,comma,sort&compress]{natbib}
\usepackage{hypernat}

% Algo XML
\usepackage{listings}

% Texte souligné
\usepackage[normalem]{ulem}

% Mise en forme des légendes

\usepackage[hang]{caption}
%\renewcommand{\captionfont}{\it}
%\renewcommand{\captionlabelfont}{\bf}
%\renewcommand{\captionlabeldelim}{$\quad$}
\usepackage{subcaption}

\newcommand{\atomes}{{\em{\bf{atomes}}}}
\newcommand{\github}{\href{https://github.com}{GitHub}}
\newcommand{\gitlab}{\href{https://gitlab.com}{GitLab}}
\newcommand{\salsa}{\href{https://salsa.debian.org}{Salsa}}
\newcommand{\email}{your.email@host.eu}
\newcommand{\metamail}{your.email\_AT\_host.eu}
\newcommand{\gitauth}{https://github.com/Author}
\newcommand{\gitprog}{\gitauth/Program}
\newcommand{\tabul}{\hspace{0.5cm}}
%\usepackage[format=plain,labelfont=bf,up,textfont=it,up]{caption}
\newcommand{\red}[1]{\textcolor{red}{#1}}
\newcommand{\pink}[1]{\textcolor{pink}{#1}}
\newcommand{\green}[1]{\textcolor{green}{#1}}
\newcommand{\cyan}[1]{\textcolor{cyan}{#1}}
\definecolor{cadmium}{rgb}{0.05,0.5,0.06}
\newcommand{\dgreen}[1]{\textcolor{cadmium}{#1}}
\newcommand{\blue}[1]{\textcolor{blue}{#1}}
\newcommand{\brown}[1]{\textcolor{brown}{#1}}
\newcommand{\dblue}[1]{\textcolor{DodgerBlue4}{#1}}
\newcommand{\aqua}[1]{\textcolor{SpringGreen3}{#1}}
\newcommand{\magenta}[1]{\textcolor{magenta}{#1}}
\newcommand{\violet}[1]{\textcolor{violet}{#1}}
\definecolor{lg}{rgb}{0.95,0.95,0.95}
\newcommand{\rtt}[1]{{\bf{\texttt{\red{#1}}}}}
\newcommand{\gtt}[1]{{\bf{\texttt{\green{#1}}}}}
\newcommand{\dgtt}[1]{{\bf{\texttt{\dgreen{#1}}}}}
\newcommand{\btt}[1]{{\bf{\texttt{\blue{#1}}}}}
\newcommand{\dbtt}[1]{{\bf{\texttt{\dblue{#1}}}}}
\newcommand{\abtt}[1]{{\bf{\texttt{\aqua{#1}}}}}
\newcommand{\dctt}[1]{{\bf{\texttt{\cyan{#1}}}}}
\newcommand{\bbtt}[1]{{\bf{\texttt{\brown{#1}}}}}
\newcommand{\rgtt}[1]{\rtt{/}{\gtt{#1}}}
\newcommand{\rbtt}[1]{\rtt{/}{\btt{#1}}}
\newcommand{\bftt}[1]{{\bf{\texttt{#1}}}}

\usepackage{placeins}
\usepackage{keystroke}

% Créer un environnement codage bash
\newsavebox{\cobox}
\def\script{
  \noindent \\[0.25cm] \\
  \begin{lrbox}
  \cobox
  \begin{minipage}[l]{16cm}
  \begin{alltt}}
\def\endscript{
  \end{alltt}
  \end{minipage}
  \end{lrbox}
  \colorbox{lg}{\usebox{\cobox}}
  \vspace{0.25cm}\par\noindent}

% Idem mais avec puces (itemize) -1.0334 cm / puces
\newsavebox{\coboxi}
\def\scripti{
  \noindent \\[0.25cm] \\
  \begin{lrbox}
  \coboxi
  \begin{minipage}[l]{14.966cm}
  \begin{alltt}}
\def\endscripti{
  \end{alltt}
  \end{minipage}
  \end{lrbox}
  \colorbox{lg}{\usebox{\coboxi}}
  \vspace{0.25cm}\par\noindent}

\newsavebox{\coboxii}
\def\scriptii{
  \noindent \\[0.25cm] \\
  \begin{lrbox}
  \coboxii
  \begin{minipage}[l]{13.9332cm}
  \begin{alltt}}
\def\endscriptii{
  \end{alltt}
  \end{minipage}
  \end{lrbox}
  \colorbox{lg}{\usebox{\coboxii}}
  \vspace{0.25cm}\par\noindent}

\newsavebox{\coboxiii}
\def\scriptiii{
  \noindent \\[0.25cm] \\
  \begin{lrbox}
  \coboxiii
  \begin{minipage}[l]{12.9004cm}
  \begin{alltt}}
\def\endscriptiii{
  \end{alltt}
  \end{minipage}
  \end{lrbox}
  \colorbox{lg}{\usebox{\coboxiii}}
  \vspace{0.25cm}\par\noindent}

\newcommand{\fscript}[1]{{\footnotesize{\begin{script}#1\end{script}}}}
\newcommand{\sscript}[1]{{\scriptsize{\begin{script}#1\end{script}}}}

%Coloration synthaxique
\newcommand{\bash}{\textcolor{blue}{\#!/bin/bash}}
\newcommand{\bif}{\textcolor{brown}{{\bf{\texttt{if [}}}}}
\newcommand{\then}{\textcolor{brown}{{\bf{\texttt{]; then}}}}}
\newcommand{\bli}{\textcolor{brown}{{\bf{\texttt{elif [}}}}}
\newcommand{\bel}{\textcolor{brown}{{\bf{\texttt{else}}}}}
\newcommand{\bfi}{\textcolor{brown}{{\bf{\texttt{fi}}}}}
\newcommand{\blet}{\textcolor{brown}{{\bf{\texttt{let}}}}}
\newcommand{\bfor}{\textcolor{brown}{{\bf{\texttt{for}}}}}
\newcommand{\while}{\textcolor{brown}{{\bf{\texttt{while [}}}}}
\newcommand{\until}{\textcolor{brown}{{\bf{\texttt{until [}}}}}
\newcommand{\bin}{\textcolor{brown}{{\bf{\texttt{in}}}}}
\newcommand{\bdo}{\textcolor{brown}{{\bf{\texttt{do}}}}}
\newcommand{\done}{\textcolor{brown}{{\bf{\texttt{done}}}}}
\newcommand{\comm}[1]{\textcolor{blue}{\# #1}}
\newcommand{\var}[1]{\textcolor{cyan}{#1}}
\newcommand{\dvar}[1]{\textcolor{violet}{\$#1}}
\newcommand{\vard}[1]{\$#1}
\newcommand{\echo}{\textcolor{brown}{{\bf{\texttt{echo}}}}}
\newcommand{\bcp}{\textcolor{brown}{{\bf{\texttt{cp}}}}}
\newcommand{\cd}{\textcolor{brown}{{\bf{\texttt{cd}}}}}
\newcommand{\mv}{\textcolor{brown}{{\bf{\texttt{mv}}}}}
\newcommand{\brm}{\textcolor{brown}{{\bf{\texttt{rm}}}}}
\newcommand{\mkdir}{\textcolor{brown}{{\bf{\texttt{mkdir}}}}}
\newcommand{\mkdirp}{\textcolor{brown}{{\bf{\texttt{mkdir -p}}}}}
\newcommand{\bnum}[1]{\textcolor{red}{\texttt{#1}}}
\newcommand{\say}[1]{\textcolor{brown}{"}\textcolor{magenta}{#1}\textcolor{brown}{"}}
\newcommand{\beq}{\textcolor{brown}{{\bf{\texttt{-eq}}}}}
\newcommand{\bne}{\textcolor{brown}{{\bf{\texttt{!=}}}}}
\newcommand{\bap}{\textcolor{brown}{\texttt{"}}}
\newcommand{\pipe}{\textcolor{brown}{{\bf{\texttt{|}}}}}
\newcommand{\svar}[1]{\textcolor{blue}{\texttt{`#1`}}}
\newcommand{\reg}[1]{\bbtt{\textquotesingle}\magenta{#1}\bbtt{\textquotesingle}}
\newcommand{\bad}[1]{\textcolor{brown}{{\bf{\texttt{#1}}}}}
\newcommand{\lint}{\bftt{lintian}}
\newcommand{\piup}{\rtt{sudo} \bftt{piuparts}}
\newcommand{\func}[1]{\textcolor{cyan}{{\bf{\texttt{function #1 \{ }}}}}
\newcommand{\efunc}{\textcolor{cyan}{{\bf{\texttt{\}}}}}}

%markdown:
\newcommand{\mdcom}[1]{\textcolor{cyan}{\texttt{<!-{-} #1 -{-}>}}}

% prompt
\newcommand{\uprompt}[1]{\dgtt{user@localhost}:\btt{#1}\$}
\newcommand{\fprompt}[1]{user@localhost #1]\$}
\newcommand{\lsr}{\bftt{ls} \rtt{-R}}

% configure
\newcommand{\bmpt}[2]{\begin{minipage}{3cm}#1\end{minipage} & $\Longrightarrow$ & 
	\begin{minipage}{12cm}#2\end{minipage} \\ \\ \\}
\newcommand{\dnl}[1]{\textcolor{blue}{dnl #1}}
\newcommand{\cpink}[1]{\textcolor{pink}{\texttt{#1}}}
\newcommand{\confa}[1]{\textcolor{cyan}{{\bf{\texttt{#1}}}}}
\newcommand{\confb}[2]{\textcolor{cyan}{{\bf{\texttt{#1(}}}}#2\textcolor{cyan}{{\bf{\texttt{)}}}}}
\newcommand{\cstr}[1]{\dgtt{[}#1\dgtt{]}}

% Command
\newcommand{\cmd}[4]{
\begin{itemize}
\item
\begin{tabular*}{0.95\textwidth}{l@{\extracolsep{\fill}}r}
\textcolor{blue}{\bf{\texttt{#1}}} & {\it{"#2"}}
\end{tabular*} \\
\hspace{1cm} - usage:\quad {\bf{\texttt{#1 #3}}} \\
\hspace{1cm} - ex:   \quad \texttt{ user@localhost ]\$ #1 #4}
\end{itemize}
}
% Command with options
\newcommand{\cmdo}[5]{
\begin{itemize}
\item
\begin{tabular*}{0.95\textwidth}{l@{\extracolsep{\fill}}r}
\textcolor{blue}{\bf{\texttt{#1}}} & {\it{"#2"}} 
\end{tabular*} \\
\hspace{1cm} - usage:\quad {\bf{\texttt{#1 #3}}} \\
\hspace{1cm} - ex:   \quad \texttt{ user@localhost ]\$ #1 #4} \\
\hspace{1cm} - options: #5
\end{itemize}
}
% Command with redirection
\newcommand{\cmdr}[4]{
\begin{itemize}
\item
\begin{tabular*}{0.95\textwidth}{l@{\extracolsep{\fill}}r}
\textcolor{blue}{\bf{\texttt{#1}}} & {\it{"#2"}}
\end{tabular*} \\
\hspace{1cm} - usage:\quad {\bf{\texttt{#3}}} \\
\hspace{1cm} - ex:   \quad \texttt{ user@localhost ]\$ #4}
\end{itemize}
}

% Custom environemnts for the tables 'tableh' and figures 'figureh'
% to deal both with font size and the centering, so that you do not need to use 
% a \begin{center} and \end{center} each time
% You can adapt it to your liking by changing the font size command \scriptsize
\def\tableh{
  \begin{table}[h!]\centering}
  \def\endtableh{\par\end{table}}

\def\figureh{
  \begin{figure}[h!]\centering}
\def\endfigureh{\par\end{figure}}

\def\tablep{
  \begin{table}[p!]\centering}
  \def\endtablep{\par\end{table}}

\def\figurep{
  \begin{figure}[p!]\centering}
\def\endfigurep{\par\end{figure}}
