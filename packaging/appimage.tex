\newpage
\newcommand{\abuild}{\bftt{appimage-builder}}
\newcommand{\pbuild}{\bftt{pkg2appimage}}

\section{Appimage}

\href{https://appimage.org/}{AppImage} is an open-source format for distributing portable software on Linux.  
It aims to allow the installation of programs independently of specific Linux distributions, a concept often referred to as upstream packaging. \\
As a result, an AppImage can be installed and run across \href{https://ubuntu.com}{Ubuntu}, \href{https://archlinux.org}{Arch Linux}, and \href{https://www.redhat.com}{Red Hat Enterprise Linux} without needing to use specific files. 
The AppImage is a format that's self-contained, rootless, and independent of the underlying Linux distribution. 

\subsection{Prerequesites}

Give a look to the following tutorials:
\begin{itemize}
\item \href{https://docs.appimage.org/}{https://docs.appimage.org/}
\item \href{https://docs.appimage.org/packaging-guide/index.html}{https://docs.appimage.org/packaging-guide/index.html}
\item \href{https://appimage-builder.readthedocs.io/en/latest/}{https://appimage-builder.readthedocs.io/en/latest/}
\item \href{https://github.com/AppImage/AppImageKit/wiki/Creating-AppImages/cc2441518975caca23e9ce2dba6f08a22c678d1e}{Creating appimage}
\item \href{https://docs.appimage.org/packaging-guide/converting-binary-packages/pkg2appimage.html}{https://docs.appimage.org/packaging-guide/converting-binary-packages/pkg2appimage.html}
\end{itemize}
\vspace{0.25cm}
To create an appimage of your program, install \abuild:
{\notsoscript{
\begin{script}
\bftt{wget} \dgtt{-O} appimage-builder-x86\_64.AppImage \textbackslash
https://github.com/AppImageCrafters/appimage-builder/releases/download/v1.1.0/appimage-builder-1.1.0-x86\_64.AppImage

# make executable
\bftt{chmod} \dgtt{+x} appimage-builder-x86\_64.AppImage

# install in PATH to make it available as command (optional)
\rtt{sudo} \bftt{mv} appimage-builder-x86\_64.AppImage /usr/local/bin/appimage-builder
\end{script}
}}
For RedHat based Linux install \bftt{apt}:
\begin{script}
\fprompt{~} sudo dnf install apt
\end{script}
\newpage
\subsection{\abuild}

To build an appimage using \abuild\ requires to prepare a recipe in a JSON file. \\
Several methods are available to let \abuild\ know how to retreive your program's data to build the appimage. 

\subsubsection*{Recipe using a local build of your program}

\begin{itemize}
\item Configure and build the program with the installation prefix set to "\texttt{/usr}"
{\footnotesize{
\begin{scripti}
\fprompt{~/program-1.2.12} \bftt{./configure} \dgtt{--prefix=/usr}
\fprompt{~/program-1.2.12} \bftt{make}
\end{scripti}
}}
\item Create an "\texttt{AppDir}" directory to install the program using the \bftt{DESTDIR} variable: 
{\footnotesize{
\begin{scripti}
\fprompt{~/program-1.2.12} \bftt{mkdir} \dgtt{AppDir}
\fprompt{~/program-1.2.12} \bftt{make} \dgtt{install} \rtt{DESTDIR}=\dvar{PWD}"/AppDir"
\end{scripti}
}}
\item Prepare you \abuild\ recipe, in a file named "\texttt{program.yml}" 
{\footnotesize{
\begin{scripti}
\comm{appimage-builder recipe, for details see:}
\comm{\href{https://appimage-builder.readthedocs.io}{https://appimage-builder.readthedocs.io}}
\yml{version} \magenta{1}
\yml{AppDir}
  \yml{path} ./AppDir
  \yml{app\_info}
    \yml{id} program     \comm{Name of the .desktop file}
    \yml{name} program   \comm{Name of the program}
    \yml{icon} program   \comm{Name of the icon file, no extention required}
    \yml{version} \magenta{1.2.12}
    \yml{exec} usr/bin/program
    \yml{exec\_args} \$@
  \yml{test}
    \yml{fedora}
      \yml{image} appimagecrafters/tests-envfedora-30
      \yml{command} ./AppRun
      \yml{use\_host\_x} True
    \yml{debian}
      \yml{image} appimagecrafters/tests-envdebian-stable
      \yml{command} ./AppRun
      \yml{use\_host\_x} True
\yml{AppImage}
  \yml{arch} x86\_64
  \yml{update-information} guess
\end{scripti}
}}
\end{itemize}

\newpage
\subsubsection*{Recipe using packages from official Debian repositories}

\begin{itemize}
\item Prepare you \abuild\ recipe, in a file named "\texttt{program.yml}" 
{\notsoscript{
\begin{scripti}
\comm{appimage-builder recipe, for details see:}
\comm{\href{https://appimage-builder.readthedocs.io}{https://appimage-builder.readthedocs.io}}
\yml{version} \magenta{1}
\yml{AppDir}
  \yml{path} ./AppDir
  \yml{app\_info}
    \yml{id} program     \comm{Name of the .desktop file}
    \yml{name} program   \comm{Name of the program}
    \yml{icon} program   \comm{Name of the icon file, no extention required}
    \yml{version} \magenta{1.2.12}
    \yml{exec} usr/bin/program 
    \yml{exec\_args} \$@
  \yml{apt}
    \yml{arch} amd64
    \yml{sources}
      \comm{One sourceline instruction for each repository to use to search for packages: }
      \comm{In this example official Debian repositories: }
      \comm{  - bookworkm: the official Debian 12 main repository}
      \comm{  - bookworkm-backports: the official Debian 12 backports main repository}
      \comm{Each repository requires to GPG signed, for some reasons the script main fails,}
      \comm{and it is declare the GPG key to use using the 'signed-by=/file-path/key-file.gpg' option:}
      \bbtt{-} \yml{sourceline} \magenta{{\textquotesingle}deb [arch=amd64, signed-by=/usr/share/keyrings/debian-archive-bookworm-automatic.gpg] 
                          http://deb.debian.org/debian bookworm main{\textquotesingle}}
      \bbtt{-} \yml{sourceline} \magenta{{\textquotesingle}deb [arch=amd64, signed-by=/usr/share/keyrings/debian-archive-bookworm-automatic.gpg] 
                          http://deb.debian.org/debian bookworm-backports main{\textquotesingle}}
    \comm{Package and dependencies to be included:}
    \yml{include}
      \bbtt{-} program
      \bbtt{-} program-data
      \bbtt{-} libgtk-3-0
      \bbtt{-} libxml2
      \bbtt{-} libpangoft2-1.0-0
      \bbtt{-} libglu1-mesa
      \bbtt{-} libepoxy0
      \bbtt{-} libavutil57
      \bbtt{-} libavcodec59
      \bbtt{-} libavformat59
      \bbtt{-} libswscale6
\yml{AppImage}
  \yml{arch} x86\_64
  \yml{update-information} guess
\end{scripti}
}}
\end{itemize}
\vspace{-0.125cm}
The \yml{apt} section describes packages to download using the \bftt{apt} command, if you are not using a Debian based Linux. \\
Repositories must be GPG signed, however \abuild\ has issues to download and install keys during the process. 
It is safer to install the key(s) before hand manualy and then declare the key to use in the recipe file using the \magenta{\texttt{signed-by=/file-path/key.gpg}} instruction. 

\subsubsection*{Build the appimage using \abuild\ and the recipe}

To build the appimpage use:
{\footnotesize{
\begin{script}
\fprompt{~/program-1.2.12} \abuild \dgtt{--recipe} \rtt{program.yml}
\end{script}
}}
\\[-0.5cm]
\abuild\ will download and install all required packages in a new "\texttt{AppDir}" directory, and will create an appimage in the active directory with the name "\rtt{program}\texttt{-}\btt{version}\texttt{-}\gtt{arch}\texttt{.AppImage}" where the values for \rtt{program}, \btt{version} and \gtt{arch} come from the corresponding fields in the recipe file. 
