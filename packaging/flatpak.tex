\section{Flatpak}

\href{https://flatpak.org}{Flatpak} is a utility for software deployment and package management for Linux. \\
It offers a sandbox environment in which users can run application software in isolation from the rest of the system. 

\subsection{Prerequisites}

Give a look to the following tutorials:
\begin{itemize}
\item \href{https://flatpak.org/setup}{https://flatpak.org/setup}
\item \href{https://docs.flatpak.org/en/latest/first-build.html}{https://docs.flatpak.org/en/latest/first-build.html}
\end{itemize}
To install Flatpak and the tools required to build your program:
\begin{itemize}
\item Install Flatpak: 
\begin{itemize}
\item Fedora: 
{\footnotesize{
\begin{scriptii}
\fprompt{~} \bftt{sudo} \rtt{dnf} \btt{install} flatpak
\end{scriptii}
}}
\item Debian:
{\footnotesize{
\begin{scriptii}
\uprompt{~} \bftt{sudo} \rtt{apt} \btt{install} flatpak
\uprompt{~} \bftt{sudo} \rtt{apt} \btt{install} gnome-software-plugin-flatpak
\end{scriptii}
}}
\end{itemize}
\item Install the Flathub repository: 
{\scriptsize{
\begin{scripti}
\bftt{flatpak remote-add} \rtt{-{-}if-not-exists} \btt{flathub} https://dl.flathub.org/repo/flathub.flatpakrepo
\end{scripti}
}}
\item Install the suitable Flatpak runtime:
{\footnotesize{
\begin{scripti}
\bftt{flatpak} \rtt{install} \rtt{flathub} \btt{org.freedesktop.Platform//22.08}
\end{scripti}
}}
\item Install the corresponding Flatpak Software Development Kit "SDK":
{\footnotesize{
\begin{scripti}
\bftt{flatpak} \rtt{install} \rtt{flathub} \btt{org.freedesktop.Sdk//22.08}
\end{scripti}
}}
\item Install the Flatpak builder:
{\footnotesize{
\begin{scripti}
\bftt{flatpak} \rtt{install} \rtt{flathub} \btt{org.flatpak.Builder}
\end{scripti}
}}
\end{itemize}

\subsection{To build the Flatpak}

\begin{enumerate}
\item Prepare the YAML file that describe your project: "\texttt{org.flatpak.prog.yml}"
\item Init a Git repository and install the shared modules repository
\begin{itemize}
\item This is required to use glu in the example
\item You might need it for other purposes later on
\end{itemize}
\item Build the Flatpak
\end{enumerate}

\subsubsection{The file "\texttt{org.flatpak.prog.yml}"}

This file describes the construction protocol to build the Flatpak for your project:
{\footnotesize{
\begin{script}
\yml{app-id} org.flatpak.prog
\yml{runtime} org.freedesktop.Platform
\yml{runtime-version} \magenta{{\textquotesingle}22.08{\textquotesingle}}
\yml{sdk} org.freedesktop.Sdk
\yml{command} prog
\yml{modules}
  \bbtt{-} shared-modules/glu/glu-9.json
  \bbtt{-} \yml{name} prog
    \yml{buildsystem} autotools
    \yml{no-autogen} \magenta{true}
    \yml{sources}
      \bbtt{-} \yml{type} archive
        \yml{url} https//github.com/\rtt{AUTHOR}/\bftt{RESPOSITORY}/archive/refs/tags/v1.2.12.tar.gz
        \yml{sha256} \magenta{"c9a540d0492f8c6b4d829b7cb8df5fc2fa5cf6374ee4e3c13f90865e9303d143"}
\end{script}
}}
\\[-0.5cm]
\noindent This file points to a release in your \github\ or \gitlab\ repository. \\
The value following "\texttt{sha256}" is the corresponding SHA256 check sum obtained using: 
\begin{script}
\bftt{sha256sum} v1.2.12.tar.gz
\end{script} \\[-0.5cm]
\noindent
The builder will download the package from your repository, and control that it is a match. 
\newpage
\noindent {\bf{Note for GTK developers (and maybe others)}}\\[0.25cm]
To distribute Flatpak requires to follow some rules for the internal name of the application set calling the "\href{https://docs.gtk.org/gtk4/ctor.Application.new.html}{\texttt{gtk\_application\_new}}" function. \\
The name should follow the syntax of the name of the appstream metadata file used to describe the package (see Sections~\ref{rpmpost} and \ref{appstream} for details). \\[0.25cm]
For example with a file named "\texttt{com.program.www.appdata.xml}":
\begin{script}
\brown{GtkApplication} * NewApp = \magenta{NULL};
\green{int} time\_id = (\green{int})clock ();
gchar * app\_name = g\_strdup\_printf (\magenta{"com.program.www.appdata-}\%d\magenta{"}, time\_id);
\cyan{#if GLIB\_MINOR\_VERSION <} \magenta{74}
  NewApp = gtk\_application\_new (app\_name, G\_APPLICATION\_FLAGS\_NONE);
\cyan{#else}
  NewApp = gtk\_application\_new (app\_name, G\_APPLICATION\_DEFAULT\_FLAGS);
\cyan{#endif}
g\_free (app\_name);
\end{script}

\subsubsection{Init a Git repository and install the shared modules repository}

\vspace{-0.25cm}
{\footnotesize{
\begin{script}
\bftt{git} \rtt{init}
\bftt{git} \rtt{submodule} \btt{add} https://github.com/flathub/shared-modules.git
\end{script}
}}
\vspace{-0.25cm}

\subsubsection{Building the Flatpak}

\vspace{-0.25cm}
{\footnotesize{
\begin{script}
\bftt{ls}
org.flatpak.prog.yml
\bftt{flatpak-builder} \rtt{prog} \btt{org.flatpak.prog.yml} \dgtt{-{-}force-clean}
\end{script}
}}
\vspace{-0.25cm}

\subsection{Running the Flatpak}

If your Flatpak is not in the official repositories your need to install it manually:
{\scriptsize{
\begin{script}
\bftt{flatpak-builder} \dgtt{-{-}user} \dgtt{-{-}install} \dgtt{-{-}force-clean} \rtt{prog} \btt{org.flatpak.prog.yml}
\end{script}
}}
\\[-0.25cm]
\noindent Then to run the Flatpak use:
{\scriptsize{
\begin{script}
\bftt{flatpak-builder} \dgtt{-{-}socket=session-bus} \dgtt{-{-}nosocket=fallback-x11} \dgtt{-{-}socket=x11} \btt{org.flatpak.prog.yml}
\end{script}
}}

